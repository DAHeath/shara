\section{Related Work}
\label{sec:related-work}
% different classes of CHC solvers:
A significant body of previous work has presented solvers for
different classes of Constrained Horn Clauses, or finding inductive
invariants of programs that correspond to solutions of CHCs.
% solving linear systems:
\impact attempts to verify a given sequential procedure, which
corresponds to solving a recursive linear CHC
system~\cite{mcmillan06}.
%
\impact attempts to verify a given procedure by iteratively selecting
paths and synthesizing invariants for each path.
%
Such an approach corresponds to iteratively and selecting and solving
derivations of a corresponding linear CHC system.

% interprocedural verification:
Previous work also proposed a verifier for recursive
programs~\cite{heizmann10}.
%
The proposed approach selects interprocedural paths of a program and
synthesizes invariants for each as nested interpolants.
%
Such an approach corresponds to attempting to solve a recursive CHC
system $\mathcal{S}$ by selecting derivation trees of $\mathcal{S}$
and solving each tree.

% solving recursive systems:
Previous work has proposed solvers for recursive systems that, given a
system $\mathcal{S}$, attempt to solve $\mathcal{S}$ by generating and
solving a series of recursion-free unwindings of $\mathcal{S}$.
%
In particular, \duality attempts to solve each unwinding
$\mathcal{S}'$ by selecting and solving derivation-trees of
$\mathcal{S}'$~\cite{bjorner13}.
%
Other optimizations selects derivation trees to solve using symbolic
symbolic analogs of Prolog evaluation with
tabling~\cite{jaffar09,mcmillan14}.
%
\eldarica attempts to solve each recursion-free CHC system by lazily
copying its subsystems to form a body-disjoint
over-approximation~\cite{rummer13a,rummer13b}.

% DAG interpolation:
\whale attempts to verify a sequential recursive program by generating
and solving hierarchical programs (i.e., programs that may contain
conditional branches and procedure calls, but do not contain loops or
recursion), which correspond to recursion-free CHC
systems~\cite{albarghouthi12b}.
%
To solve a particular recursion-free system $\mathcal{S}$, \whale
generates a linear inlining $\mathcal{S}'$ of $\mathcal{S}$ and solves
it using a procedure \vinta~\cite{albarghouthi12a}.
%
In general, $\mathcal{S}'$ may have size exponential in the size of
$\mathcal{S}$.

% compare sys to all of the previous work:
\sys is similar to all of the approaches given above for solving
recursion-free CHC systems in that it reduces the problem of solving a
given recursion-free CHC system $\mathcal{S}$ to solving a CHC system
in a directly-solvable class.
%
\sys is distinct from all of the approaches given above in that it
reduces solving a recursion-free CHC system to solving a
Clause-Dependent Disjoint (CDD) system.
%
CDD systems strictly contain the union of all classes of
directly-solvable CHC systems used by the above approaches, and can
themselves be solved directly.

% bullshit CADE papers:
Previous work has given solvers for non-linear Horn clauses over
particular theories.
%
In particular, verifiers have been proposed for recursion-free systems
over the theory of linear arithmetic~\cite{komuravelli14}.
%
Because the verifier relies on quantifier elimination, it is not clear
if it can be extended to richer theories that support interpolation,
such as the combination of linear arithmetic with uninterpreted
functions.
%
Other work gives a solver for the class of \emph{timed pushdown
  systems}, a subclass of CHC systems over the theory of linear real
arithmetic~\cite{hoder12}.
%
Unlike both approaches, \sys can solve systems over any theory that
supports interpolation.

% DAG inlining:
DAG inlining, given a hierarchical program $P$, attempts to generate a
compact verification condition for $P$~\cite{lal-qadeer15}.
%
\sys contains a procedure that, given a recursion-free Horn Clause
system $\mathcal{S}$, attempts to construct a compact CDD system
$\mathcal{S}'$ such that each solution of $\mathcal{S}'$ defines a
solution of $\mathcal{S}$.
%
Because hierarchical programs and recursion-free Horn Clauses
correspond closely to each other, algorithms that operate on
hierarchical programs directly correspond to algorithms that operate
on recursion-free Horn Clauses.
%
However, an algorithm for constructing a verification condition of
hierarchical programs cannot apparently be directly used to synthesize
a solution of a recursion-free CHC system.

%%% Local Variables: 
%%% mode: latex
%%% TeX-master: "p"
%%% End: 
