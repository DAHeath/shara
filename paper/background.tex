% technical background material:
\section{Background}
\label{sec:background}

% interpolants:
\subsection{Logical interpolation}
\label{sec:itps}
%
\sys solves CHC systems, logic-programming problems in which a
solution maps predicate symbols to first-order formulas.
%
All objects defined in this paper are defined over a fixed space of
first-order logical variables $X$.
%
In particular, for theory $\mathcal{T}$, the space of $\mathcal{T}$
formulas over $X$ is denoted $\tformulas{\mathcal{T}}$.
%
For each formula $\varphi \in \tformulas{\mathcal{T}}$, the set of
variables that occur in $\varphi$ (i.e., the \emph{vocabulary} of
$\varphi$) is denoted $\vocab(\varphi)$.
% define models, satisfaction, entailment
For formulas $\varphi_0, \ldots, \varphi_n, \varphi \in
\tformulas{\mathcal{T}}$, the fact that $\varphi_0, \ldots, \varphi_n$
\emph{entail} $\varphi$ is denoted $\varphi_0, \ldots, \varphi_n
\entails \varphi$.

% Define replacement and substitution:
For sequences of variables $Y = [ y_0, \ldots, y_n ]$ and $Z = [ z_0,
\ldots, z_n ]$, the $\mathcal{T}$ formula constraining the equality of
each element in $Y$ with its corresponding element in $Z$, i.e., the
formula $\bigland_{0 \leq i \leq n} y_i = z_i$, is denoted $Y = Z$.
%
The repeated replacement of variables $\varphi[ z_0 / y_0 \ldots z_{n}
/ y_{n} ]$ is denoted $\replace{\varphi}{Y}{Z}$.
%
For each formula $\varphi$ defined over free variables $Y$,
$\replace{\varphi}{Y}{Z}$ is denoted alternatively as $\varphi[ Z ]$.
%
The number of free variables in formula $\varphi$ is denoted
$\degreeof{\varphi}$.

% introduce interpolation:
An interpolant of mutually inconsistent $\mathcal{T}$ formulas
$\varphi_0$ and $\varphi_1$ is a $\mathcal{T}$ formula in their common
vocabulary that explains their inconsistency.
%
\begin{defn}
  \label{defn:itps}
  % define:
  For $\varphi_0, \varphi_1 \in \tformulas{\mathcal{T}}$, $I \in
  \tformulas{\mathcal{T}}$ such that
  %
  \textbf{(1)} $\varphi_0 \entails I$, %
  \textbf{(2)} $I, \varphi_1 \entails \false$, and %
  \textbf{(3)} $\vocab(I) \subseteq \vocab(\varphi_0) \intersection
  \vocab(\varphi_1)$,
  %
  $I$ is an \emph{interpolant} of $\varphi_0$ and $\varphi_1$.
\end{defn}
%
All spaces of formulas in the remainder of this paper will be defined
for a fixed, arbitrary theory $\mathcal{T}$ that supports
interpolation, such as combinations of the theories of linear
arithmetic and the theory of uninterpreted functions with equality.
% introduce decision procedure:
Although determining the satisfiability of formulas in such theories
is NP-complete in general, decision procedures~\cite{moura08} and
interpolating theorem provers~\cite{mcmillan04} for such theories have
been proposed that operate on such formulas efficiently.
%
\sys is defined as using a decision procedure for $\mathcal{T}$ named
\issat, and an interpolating theorem prover for $\mathcal{T}$ named
$\solveitp$.

% define CHC's:
\subsection{Constrained Horn Clauses}
\label{sec:chcs}

\subsubsection{Structure}
% definition of CHC
A Constrained Horn Clause contains a body and a head.
%
A body consisting of a conjunctive set of uninterpreted relational predicates applied to varialbes and a formula
that over variables.
%
A head can either be an uninterpreted relational predicates applied to variables or \cc{False}.
%
A Clause where the head is \cc{False} is called query.
\begin{defn}
  A Constrained Horn Clause is definied as follows: \\
  $\cc{chc} \Coloneqq \cc{head} \gets \cc{body}$ \\
  $\cc{head} \Coloneqq \cc{False} \mid \cc{pred}$  \\
  $\cc{body} \Coloneqq \varphi \wedge \cc{preds}$ \\
  $\cc{preds} \Coloneqq \cc{True} \mid \cc{pred} \wedge \cc{preds}$ \\
  $\cc{pred} \Coloneqq$ an uninterpreted predicate applied to variables \\
  $\varphi \Coloneqq$ a formula that all variables are vocabulary \\
\end{defn}
%
% define a body function, head function, and recursion-free
For a given constrained horn clause $C$, $\bodyof{C}$ denotes
the vector of uninterpreted predicates in the body of $C$, $\consof{C}$
denotes the formula in the body.
%
It is possible for a uninterpreted predicate shows multiple times in the
body of $C$.
%
For a given constrained horn clause $C$ that is not a query,
$\headof{c}$ denotes the uninterpreted predicates of the head of $C$.
%

%define a CHC system
A Constrained Horn Clauses system is a set of constrained horn clauses 
that it has one and only one query.
%
For a given Constrained Horn Clauses system $S$,$\predof{S}$ denotes 
the set of all uninterpreted predicates in this CHC system.

% define dependency predicates
\begin{defn}
  Given a CHC system $\cc{S}$ and two uninterpreted predicates $\cc{P}$ and
  $\cc{Q} \in \predof{S}$, if $\exists \cc{C} \in \cc{S}$ such that $\cc{P} = \headof{C}$
  and $\cc{Q} \in \bodyof{C}$, then $\cc{Q}$ is a dependency predicate of $\cc{P}$.
\end{defn}
%
\begin{ex}
  Because uninterpreted predicates \cc{L4} is in the body of clause (4) and uninterpreted predicates \cc{L8}
  is the head of clause (4), \cc{L4} is a dependency predicate of \cc{L8}.\autoref{chc}
\end{ex}
%
Given a CHC system $\cc{S}$ and an uninterpreted predicate $\cc{P}$, $\depsof{P}$
denotes the set of all dependency predicates of $\cc{P}$ in $\cc{S}$.
%

%define transitive dependency predicates
\begin{defn}
  Given a CHC system $\cc{S}$ and three uninterpreted predicates $\cc{P},
  \cc{Q}$ and $\cc{R} \in \predof{S}$, if $\cc{Q} \in \depsof{P}$ then \cc{Q} is a transitive dependency
  predicate of \cc{P}. 
  %
  if \cc{Q} is a transitive dependency predicate of \cc{P}, and \cc{R} is a transitive dependency predicate
  of \cc{Q},then $\cc{R}$ is a transitive dependency predicate of $\cc{P}$.
\end{defn}
%
\begin{ex}
  Because uninterpreted predicates \cc{L4} is a dependency predicate of \cc{L8}, 
  \cc{L4} is a transitive dependency predicate of \cc{L8}.
  %
  And because \cc{L8} is an transitive dependency predicate of \cc{L9}, 
  \cc{L4} is a transitive dependency predicate of \cc{L9}.\autoref{chc}
\end{ex}
%
Given a CHC system $\cc{S}$ and an uninterpreted predicate $\cc{P}$, $\tdepsof{P}$
denotes the set of all transitive dependency predicates of $\cc{P}$ in $\cc{S}$.
%

%define siblings
\begin{defn}
  Given a CHC system $\cc{S}$ and two uninterpreted predicates $\cc{P}$ and
  $\cc{Q} \in \predof{S}$, if $\exists \cc{C} \in \cc{S}$ such that $\cc{P} \in \bodyof{C}$
  and $\cc{Q} \in \bodyof{C}$, then $\cc{Q}$ and $\cc{P}$ are siblings of each other.
\end{defn}
%
\begin{ex}
  Because uninterpreted predicates \cc{L9} and \cc{db} both shows in the body of
  clause (6), \cc{L9} and \cc{db} are siblings of each other.
  %
\end{ex}
%
Given a CHC system $\cc{S}$ and an uninterpreted predicate $\cc{P}$, $\siblingof{P}$
denotes the set of all siblings predicates of $\cc{P}$ in $\cc{S}$.
%



% define system of CHC's
We will present \sys as a solver for recursion-free CHCs.
%
For a given CHC system $\cc{S}$, there is no uninterpreted predicates $\cc{P} \in \predof{S}$ 
such that $\cc{P} \in \tdepsof{P}$, then $\cc{S}$ is a recursion-free CHCs system.

%
A solution to a CHC system $S$ is an interpretation of each
uninterpreted predicate $\cc{Q} \in \predof{S}$ that each clause holds
after substitute all predicates by its interpretation.

%%% Local Variables: 
%%% mode: latex
%%% TeX-master: "p"
%%% End: 
