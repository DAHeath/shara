% technical background material:
\section{Background}
\label{sec:background}

% define CHC's:
\subsection{Constrained Horn Clauses}
\label{sec:chcs}

\subsubsection{Structure}
% definition of CHC
A Constrained Horn Clause is a logical implication where the
antecedent is called the body and the consequent is called the head.
%
The body is a conjunction of a logical formula, called the fact,
and a vector of uninterpreted predicates. The fact is an arbitrary
formula in some background logic, such as linear integer arithmetic.
%
The uninterpreted predicates are applied to variables which may or may
not appear in the fact.
%
A head can be either an uninterpreted predicate applied to variables or $False$.
%
A clause where the head is $False$ is called a query. A CHC can be
defined structurally:
\begin{align*}
\cc{chc} \Coloneqq&~\cc{head} \gets \cc{body} \\
\cc{head} \Coloneqq&~False \\
  \mid&~\cc{pred} \\
\cc{body} \Coloneqq&~\varphi \wedge \cc{preds} \\
\cc{preds} \Coloneqq&~True \\
  \mid&~\cc{pred} \wedge \cc{preds} \\
\cc{pred} \Coloneqq&~\textsf{\emph{an uninterpreted predicate applied to variables}} \\
\varphi \Coloneqq&~\textsf{\emph{a formula}} \\
\end{align*}
%
% define a body function, head function, and recursion-free
For a given CHC $C$, $\bodyof{C}$ denotes the vector of uninterpreted
predicates in the body and $\consof{C}$ denotes the fact in the
body.
%
If $C$ is not a query, then $\headof{C}$ denotes the uninterpreted
predicate in the head.
%
%define a CHC system
A CHC system is a set of CHCs where exactly one clause is a query.
%
For a given CHC system $S$, $\predof{S}$ denotes the set of all
uninterpreted predicates.

To explain the structure of a CDD system, we need terminology that relates
predicates in a CHC system including the terms \emph{predicate dependency},
\emph{transitive predicate dependency}, and \emph{sibling}.
% define dependency predicates
\begin{defn}
  Given a CHC system $\cc{S}$ and two uninterpreted predicates
  $\cc{P}$ and $\cc{Q} \in \predof{S}$, if $\exists \cc{C} \in \cc{S}$
  such that $\cc{P} = \headof{C}$ and $\cc{Q} \in \bodyof{C}$, then
  $\cc{Q}$ is a predicate dependency of $\cc{P}$.
\end{defn}
%
\begin{ex}
  In $\mcchc$, because $\cc{L}_4$ is in the body of clause (4) and
  $\cc{L}_8$ is the head of clause (4), $\cc{L}_4$ is a predicate
  dependency of $\cc{L}_8$.
\end{ex}
%
Given a CHC system $\cc{S}$ and an uninterpreted predicate $\cc{P}$,
$\depsof{P}$ denotes the set of all predicate dependencies of $\cc{P}$
in $\cc{S}$.
%
%define transitive dependency predicates
\begin{defn}
  Given a CHC system $\cc{S}$ and three uninterpreted predicates
  $\cc{P}, \cc{Q}$ and $\cc{R} \in \predof{S}$, if $\cc{Q} \in
  \depsof{P}$ then \cc{Q} is a transitive predicate dependency of
  \cc{P}.
  %
  If \cc{Q} is a transitive predicate dependency of \cc{P} and \cc{R}
  is a transitive predicate dependency of \cc{Q}, then $\cc{R}$ is a
  transitive predicate dependency of $\cc{P}$.
\end{defn}
%
\begin{ex}
  In $\mcchc$, because $\cc{L}_4$ is a predicate dependency of
  $\cc{L}_8$, $\cc{L}_4$ is a transitive predicate dependency of
  $\cc{L}_8$.
  %
  And because $\cc{L}_8$ is an transitive predicate dependency of
  $\cc{L}_9$, $\cc{L}_4$ is a transitive predicate dependency of
  $\cc{L}_9$.
\end{ex}
%
Given a CHC system $\cc{S}$ and an uninterpreted predicate $\cc{P}$,
$\tdepsof{P}$ denotes the set of all transitive predicate dependencies
of $\cc{P}$ in $\cc{S}$.
%
%define siblings
\begin{defn}
  Given a CHC system $\cc{S}$ and two uninterpreted predicates
  $\cc{P}$ and $\cc{Q} \in \predof{S}$, if $\exists \cc{C} \in \cc{S}$
  such that $\cc{P} \in \bodyof{C}$ and $\cc{Q} \in \bodyof{C}$, then
  $\cc{Q}$ and $\cc{P}$ are siblings.
\end{defn}
%
\begin{ex}
  Because uninterpreted predicates $\cc{L}_9$ and $\cc{db}$ both
  appear in the body of clause (7), $\cc{L}_9$ and $\cc{db}$ are
  siblings.
  %
\end{ex}
%
Given a CHC system $\cc{S}$ and an uninterpreted predicate $\cc{P}$,
$\siblingof{P}$ denotes the set of all siblings of $\cc{P}$
in $\cc{S}$.
%
For a given CHC system $\cc{S}$, if there are no uninterpreted
predicate $\cc{P} \in \predof{S}$ such that $\cc{P} \in \tdepsof{P}$,
then $\cc{S}$ is a \emph{recursion-free} CHCs system.
%

A solution to a CHC system $S$ is a map from each uninterpreted
predicate $\cc{P} \in \predof{S}$ to its corresponding interpretation
which is a formula.
%
For a solution to be valid, each clause in $S$ must be valid after
substituting each uninterpreted predicate by its interpretation.

% interpolants:
\subsection{Logical interpolation}
\label{sec:itps}
%
All logical objects in this paper are defined over a fixed space of first-order
variables, $X$.
%
For a theory $\mathcal{T}$, the space of $\mathcal{T}$
formulas over $X$ is denoted $\tformulas{\mathcal{T}}$.
%
For each formula $\varphi \in \tformulas{\mathcal{T}}$, the set of
variables that occur in $\varphi$ (i.e., the \emph{vocabulary} of
$\varphi$) is denoted $\vocab(\varphi)$.
%
% define models, satisfaction, entailment
For formulas $\varphi_0, \ldots, \varphi_n, \varphi \in
\tformulas{\mathcal{T}}$, the fact that $\varphi_0, \ldots, \varphi_n$
\emph{entail} $\varphi$ is denoted $\varphi_0, \ldots, \varphi_n
\entails \varphi$.

% Define replacement and substitution:
For sequences of variables $Y = [ y_0, \ldots, y_n ]$ and $Z = [ z_0,
\ldots, z_n ]$, the $\mathcal{T}$ formula constraining the equality of
each element in $Y$ with its corresponding element in $Z$, i.e., the
formula $\bigland_{0 \leq i \leq n} y_i = z_i$, is denoted $Y = Z$.
%
The repeated replacement of variables $\varphi[ z_0 / y_0 \ldots z_{n}
/ y_{n} ]$ is denoted $\replace{\varphi}{Y}{Z}$.
%
For each formula $\varphi$ defined over free variables $Y$,
$\replace{\varphi}{Y}{Z}$ is denoted alternatively as $\varphi[ Z ]$.
%
The number of free variables in formula $\varphi$ is denoted
$\degreeof{\varphi}$.

% introduce interpolation:
An interpolant of mutually inconsistent $\mathcal{T}$ formulas
$\varphi_0$ and $\varphi_1$ is a $\mathcal{T}$ formula in their common
vocabulary that explains their inconsistency.
%
\begin{defn}
  \label{defn:itps}
  % define:
  For $\varphi_0, \varphi_1, I \in \tformulas{\mathcal{T}}$, if
  %
  \textbf{(1)} $\varphi_0 \entails I$, %
  \textbf{(2)} $I \land \varphi_1 \entails \false$, and %
  \textbf{(3)} $\vocab(I) \subseteq \vocab(\varphi_0) \intersection
  \vocab(\varphi_1)$,
  %
  then $I$ is an \emph{interpolant} of $\varphi_0$ and $\varphi_1$.
\end{defn}
%
For the remainder of this paper, all spaces of formulas will be
defined for a fixed, arbitrary theory $\mathcal{T}$ that supports
interpolation, such as combinations of the theories of linear
arithmetic and the theory of uninterpreted functions with equality.
% introduce decision procedure:
Although determining the satisfiability of formulas in such theories
is NP-complete in general, decision procedures~\cite{moura08} and
interpolating theorem provers~\cite{mcmillan04} for such theories have
been proposed that operate on such formulas efficiently.
%
We define \sys in terms of an abstract decision procedure for
$\mathcal{T}$ named \issat, and an abstract interpolating theorem
prover for $\mathcal{T}$ named $\solveitp$.
