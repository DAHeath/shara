\section{Generating a Minimal CDD Expansion}
\label{app:cons-cdd}
% algorithm for constructing a CDD system:
\begin{algorithm}[t]
  % Declare IO markers.
  \SetKwInOut{Input}{Input}
  %
  \SetKwInOut{Output}{Output}
  % Declare sub-program (procedure) markers.
  \SetKwProg{myproc}{Procedure}{}{}
  % Inputs: a heap program and an error location.
  \Input{For $\mathcal{R} \in \relpreds$, %
    $\mathcal{S} \in \chc{\mathcal{R}}$.}
  % Output: inductive invariants.
  \Output{A minimal CDD expansion of $\mathcal{S}$ and
    correspondence to $\mathcal{S}$.} %
  \myproc{$\expand{ \mathcal{R} }(\mathcal{S})$ %
    \label{line:expand-begin} }{ %
    % auxiliary procedure:
    \myproc{$\expandaux( \mathcal{S}' )$
      \label{line:expand-aux-begin} }{ %
      % case split: optional sharing clause:
      \Switch{$\sharingclause( \mathcal{S}' )$}{ %
        % case: no siblings with overlapping descendants
        \lCase{$\none$:}{ \Return{$\mathcal{ S }'$} %
          \label{line:expand-ret} } %
        % case: siblings with 
        \lCase{$\mathcal{C} \in \mathcal{S}', % 
          R \in \ctxs{ \mathcal{R} }$: }{ %
          % expand the return:
          \Return{$\expandaux( %
            \copyrel{ \mathcal{R} }( %
            \mathcal{S}', \mathcal{C}, R))$ } %
          \label{line:expand-recurse}
        } %
      } %
    \label{line:expand-aux-end} } %
    % base call: 
    \Return{$( \expandaux(\mathcal{S}), \corr{ \mathcal{R} } )$ %
      \label{line:expand-base-call}}
  } %
  %
  \caption{$\expand{ \mathcal{R} }$: for $\mathcal{R} \in \relpreds$
    given $\mathcal{S} \in \chcs{ \mathcal{R} }$, returns a minimal
    CDD expansion of $\mathcal{S}$. }
  \label{alg:expand}
\end{algorithm}
% walk through top-level procedure:
\autoref{alg:expand}, given $\mathcal{S} \in \chcs{ \mathcal{R} }$
(\autoref{line:expand-begin}), returns an expansion $\mathcal{S}'$ of
$\mathcal{S}$ over relational predicates $\ctxs{ \mathcal{R} }$
(defined below) accompanied by a correspondence from $\mathcal{S}'$ to
$\mathcal{S}$.
%
$\expand{ \mathcal{R} }$ defines a procedure $\expandaux$
(\autoref{line:expand-aux-begin}---\autoref{line:expand-aux-end}) that
takes a CHC system $\mathcal{S}$ and returns an expansion of
$\mathcal{S}$.
%
$\expand{ \mathcal{R} }$ runs $\expandaux$ on $\mathcal{S}$ and
returns the result, paired with the map $\corr{ \mathcal{R} } : \ctxs{
  \mathcal{R} } \to \mathcal{R}$ (\autoref{line:expand-base-call};
$\corr{\mathcal{R}}$ is defined below).

% walk through aux procedure:
$\expandaux$, given $\mathcal{S}' \in \chcs{ \ctxs{ \mathcal{R} } }$,
runs a procedure $\sharingclause$ on $\mathcal{S}'$, which tries to
find $\mathcal{C} \in \mathcal{S}'$ and $R \in \ctxs{ \mathcal{R} }$
applied in the body of $\mathcal{C}$ that is a transitive dependences of
clause siblings in $\mathcal{S}'$.
%
I.e., $\expandaux$ tries to find $\mathcal{C}$ and $R$ such that for
some siblings $R_0, R_1 \in \ctxs{ \mathcal{R} }$, $R \in \depsof{
  \mathcal{S}' }(R_0) \intersection \depsof{ \mathcal{S}' }(R_1)$ and
$\relof{ \headof{ \mathcal{C} }} \in \depsof{ \mathcal{S}' }(R_0)
\setminus \depsof{ \mathcal{S}' }(R_1)$.
%
In such a case, we say that $(\mathcal{C}, R)$ is a
\emph{sibling-shared dependence}.
% case: no sibling-shares
If $\sharingclause$ determines that no sibling-shared dependence
exists, then $\expandaux$ returns $\mathcal{S}'$
(\autoref{line:expand-ret}).

% case: there is a sibling-shared dependence:
Otherwise, for sibling-shared dependence $(\mathcal{C}, R)$ returned
by $\sharingclause$, $\expandaux$ runs $\copyrel{ \mathcal{R} }$ on
$\mathcal{S}'$, $\mathcal{C}$, and $R$, which returns an expansion of
$\mathcal{S}'$ in which the dependents shared by $R_0$ and $R_1$ are
those shared in $\mathcal{S}'$, without $(\mathcal{C}, R)$.
%
$\expandaux$ recurses on the CHC system returned by $\copyrel{
  \mathcal{R} }$ and returns the result
(\autoref{line:expand-recurse}).

% give the representation of ctxs
\subsubsection{Representation of $\ctxs{ \mathcal{R} }$}
%
$\ctxs{ \mathcal{R} }$ are relational predicates that store the
clauses in which they occur.
%
\begin{defn}
  \label{defn:ctxs}
  $\ctxs{ \mathcal{R} }$ is the smallest space such that $\mathcal{R}
  \subseteq \ctxs{ \mathcal{R} }$ and for $\mathcal{C} \in \clauses{
    \mathcal{ \ctxs{R} } }$ and $A \in \appsof{ \bodyof{ \mathcal{C} }
  }$, $(\mathcal{C}, \relof{A}) \in \ctxs{ \mathcal{R} }$.
\end{defn}
% give the correspondence between relational predicates:
$\corr{ \mathcal{R} } : \ctxs{R} \to \mathcal{R}$ is such that for $R
\in \mathcal{R}$, $\corr{ \mathcal{R} }(R) = R$ and for $\mathcal{C}
\in \clauses{ \ctxs{R} }$ and $R \in \ctxs{R}$ such that
$(\mathcal{C}, R) \in \ctxs{R}$, $\corr{\mathcal{R}}(\mathcal{C}, R) =
\corr{ \mathcal{R} }(R)$.

% give the function that copies instances of a relational predicate
\subsubsection{Implementation of $\copyrel{ \mathcal{R} }$}
%
$\copyrel{\mathcal{R}}$, given $\mathcal{S} \in \chcs{ \mathcal{R} }$,
$\mathcal{C} \in \mathcal{S}$, and $R \in \ctxs{ \mathcal{R} }$,
returns the following CHC system.
% clause: C, but the application now points to (C, A)
Let $\mathcal{C}' \in \clauses{ \ctxs{ \mathcal{R} } }$ be
$\mathcal{C}$, with $R$ replaced with $(\mathcal{C}, R)$ in the body
of $\mathcal{C}$.
%
Let $\mathcal{S}_{ \mathcal{C} } \subseteq \clauses{ \ctxs{
    \mathcal{R} } }$ be the clauses in $\mathcal{S}$ for which $R$ is
the relational predicate of the clause head, and let
$\mathcal{S}_{\mathcal{C}}'$ be each clause in $\mathcal{S}_{
  \mathcal{C} }$, with $R$ in the clause head replaced with
$(\mathcal{C}, R)$.
%
Then $\copyrel{ \mathcal{R} }$ returns $\add{ (\remove{ \mathcal{S} }{
    \add{ \mathcal{S}_{\mathcal{C}} }{\mathcal{C} } }) }{ \mathcal{C}'
} \union \mathcal{S}_{\mathcal{C}}'$.
%  

% CDD systems:
$\expand{\mathcal{R}}$ always returns a CDD expansion of its input
(see \autoref{app:corr}, \autoref{lem:expand-corr}) that is minimal.
% discussion: there are variations:
$\expand{ \mathcal{R} }$ is certainly not unique as an algorithm for
generating a minimal CDD expansion.
%
In particular, feasible variations of $\expand{ \mathcal{R} }$ can be
generated from different implementations of $\sharingclause$, each of
which chooses clause-relations pairs to return based on different
heuristics.
%
We expect that other expansion algorithms can also be developed by
generalizing algorithms introduced in previous work on generating
compact verification conditions of hierarchical
programs~\cite{lal-qadeer15}.

\section{Proof of characterization of CDD systems}
\label{app:char}
The following is a proof of \autoref{thm:cdd-contains}.
%
\begin{proof}
  % proof structure:
  To prove that the union of the classes of linear systems and
  body-disjoint systems are strictly contained by the class of CDD
  systems, we prove %
  \textbf{(1)} the class of linear systems are contained by the class
  of CDD systems, %
  \textbf{(2)} the class of body-disjoint systems are contained by the
  class of CDD systems, and %
  \textbf{(3)} there is some CDD system that is not linear or
  body-disjoint.

  % contains linear systems:
  For goal \textbf{(1)}, let $\mathcal{S}$ be an arbitrary linear
  system.
  %
  $\mathcal{S}$ is CDD if for each clause $\mathcal{C}$ in
  $\mathcal{S}$, for each pair of distinct relational predicates in
  the body of $\mathcal{C}$, the transitive dependence are disjoint
  (\autoref{defn:cdds}).
  %
  Let $\mathcal{C}$ be an arbitrary clause in $\mathcal{S}$.
  %
  $\mathcal{C}$ has no pairs of distinct clauses, by the fact that
  $\mathcal{S}$ is linear.
  %
  Therefore, $\mathcal{S}$ is CDD.

  % contains body-disjoint systems:
  For goal \textbf{(2)}, let $\mathcal{S}$ be an arbitrary
  body-disjoint system.
  %
  The dependence relation of $\mathcal{S}$ is a tree $T$, by the
  definition of a body-disjoint system.
  %
  Let $\mathcal{C}$ be an arbitrary clause in $\mathcal{S}$, with
  distinct relational predicates $R_0$ and $R_1$ in its body.
  %
  All dependences of $R_0$ and $R_1$ are in subtrees of $T$, which
  are disjoint by the definition of a tree.
  %
  Thus, $\mathcal{S}$ is CDD, by \autoref{defn:cdds}.

  % distinguishing example:
  For goal \textbf{(3)}, the system $\mcchc$ is CDD, but is not linear
  or body-disjoint.
\end{proof}

\section{Proof of Correctness}
\label{app:corr}
%
In this section, we give a proof that \sys is a correct solver for
recursion-free CHC systems.
%
We first establish lemmas for the correctness of each procedure used
by \sys, namely \textsc{Collapse} (\autoref{lem:expansion-sound} and
\autoref{lem:expansion-complete}), $\expand{\mathcal{R}}$
(\autoref{lem:expand-corr}), and $\solvecdd{ \mathcal{R} }$
(\autoref{lem:vc}, \autoref{lem:cdd-soln-sound}, and
\autoref{lem:cdd-soln-complete}).
%
We combine the lemmas to prove correctness of \sys
(\autoref{thm:corr}).

% lem: result connecting expansions to constructing solutions
For CHC systems $\mathcal{S}$ and $\mathcal{S}'$, if $\mathcal{S}'$ is
an expansion of $\mathcal{S}$, then the result of collapsing a
solution of $\mathcal{S}'$ is a solution of $\mathcal{S}$.
%
\begin{lem}
  \label{lem:expansion-sound}
  For $\mathcal{R}, \mathcal{R}' \in \relpreds$, $\mathcal{S} \in
  \chcs{ \mathcal{R} }$, $\mathcal{S}' \in \chcs{ \mathcal{R}' }$,
  $\sigma' \in \interps{ \mathcal{R}' }$, and $\eta : \mathcal{R}' \to
  \mathcal{R}$ a correspondence from $\mathcal{S}'$ to $\mathcal{S}$,
  $\collapse{\eta}{\sigma'}$ is a solution of $\mathcal{S}$.
\end{lem}
%
\begin{proof}
  % introduce 
  Let $\mathcal{B} \in \bodies{\mathcal{R}}$, $\varphi \in \formulas$,
  and $R \in \mathcal{R}$ be such that $((\mathcal{B}, \varphi),
  R(\vars{ \mathcal{S} }(R))) \in \mathcal{S}$.
  %
  For each $R' \in \mathcal{R}'$ such that $\eta(R') = R$, there is
  some $\mathcal{B}' \in \bodies{ \mathcal{R}' }$ such that
  $((\mathcal{B}', \varphi), R'(\vars{ \mathcal{S} }(R))) \in
  \mathcal{S}'$, by the fact that $\mathcal{S}'$ is an expansion of
  $\mathcal{S}$ and the definition of an expansion.
  %
  $\sigma'(\mathcal{B}'), \varphi \entails \sigma'(\mathcal{R}')$ by
  the fact that $\sigma'$ is a solution of $\mathcal{S}'$.
  %
  $\collapse{ \eta }{ \sigma' }(\mathcal{B}), \varphi \entails
  \sigma'(\mathcal{R}')$, by the definition of
  $\collapse{\eta}{\sigma'}$.
  %
  Therefore,
  %
  \[ \collapse{ \eta }{ \sigma' }( \mathcal{B} ), \varphi \entails %
  \left( \bigland_{ \substack{R' \in \mathcal{R}' \\ \eta(R') = R} }
  \sigma(R') \right) = %
  \collapse{\eta}{\sigma'}(\mathcal{R}') 
  \]
  % 
  Therefore, $\collapse{\eta}{\sigma'}$ is a solution of clause
  $((\mathcal{B}, \varphi), R(\varsof{ \mathcal{S} }(R)))$.
  %
  Therefore, $\collapse{\eta}{\sigma'}$ is a solution of
  $\mathcal{S}$.
\end{proof}

% completeness of expansions:
For CHC solvable system $\mathcal{S}$, each expansion of $\mathcal{S}$
is solvable.
\begin{lem}
  \label{lem:expansion-complete}
  For $\mathcal{R}, \mathcal{R}' \in \relpreds$, $\mathcal{S} \in
  \chcs{ \mathcal{R} }$ such that $\mathcal{S}$ is solvable, and
  $\mathcal{S}' \in \chcs{ \mathcal{R}' }$ such that $\mathcal{S'}$ is
  an expansion of $\mathcal{S}$, $\mathcal{S}'$ is solvable.
\end{lem}
%
\begin{proof}
  Let $\sigma : \mathcal{R} \to \formulas$ be a solution of
  $\mathcal{S}$, and let $\eta : \mathcal{R}' \to \mathcal{R}$ be a
  correspondence from $\mathcal{R}'$ to $\mathcal{R}$.
  %
  Let $\sigma' : \mathcal{R}' \to \formulas$ be such that for each $R'
  \in \mathcal{R}'$, $\sigma'(R') = \sigma(\eta(R'))$.
  %
  Then $\sigma$ is a solution of $\mathcal{S}'$.
\end{proof}

% CDD systems:
$\expand{\mathcal{R}}$ always returns a CDD expansion of its input.
%
\begin{lem}
  \label{lem:expand-corr}
  For $\mathcal{R} \in \relpreds$ and $\mathcal{S} \in \chcs{
    \mathcal{R} }$, $\mathcal{S} \in \chcs{ \ctxs{ \mathcal{R} } }$,
  and $\eta : \ctxs{ \mathcal{R} } \to \mathcal{R}$ such that
  $(\mathcal{S}', \eta) = \expand{\mathcal{R}}(\mathcal{S})$, $\eta$
  is a correspondence from $\mathcal{S}'$ to $\mathcal{S}$.
\end{lem}
%
\begin{proof}
  Proof by induction on the evaluation of $\expand{ \mathcal{R} }$ on
  an arbitrary $\mathcal{S} \in \chcs{ \mathcal{R} }$.
  % inductive fact:
  The inductive fact is that in each step of evaluation \expandaux,
  $\corr{ \mathcal{R} }$ is a correspondence from argument
  $\mathcal{S}'$ to $\mathcal{S}$.
  % base case:
  For the base case, \expandaux is called initially on $\mathcal{S}$,
  by \autoref{alg:expand}.
  % 
  $\corr{ \mathcal{R} }$ is a correspondence from $\mathcal{S}$ to
  itself, by the definition of $\corr{ \mathcal{R} }$
  (\autoref{app:cons-cdd}).

  % inductive case:
  For the inductive case, for $\mathcal{C} \in \mathcal{S}$ and $R \in
  \ctxs{ \mathcal{R} }$, $\expandaux$ calls itself on $\copyrel{
    \mathcal{R} }(\mathcal{S}, \mathcal{C}, R)$.
  %
  For each $\mathcal{S}' \in \chcs{ \ctxs{ \mathcal{R} } }$ such that
  $\corr{ \mathcal{R} }$ is a correspondence from $\mathcal{S}'$ to
  $\mathcal{S}$, $\corr{ \mathcal{R} }$ is a correspondence from
  $\copyrel{ \mathcal{R} }( \mathcal{S}', \mathcal{C}, R)$ to
  $\mathcal{S}$, by definition of $\copyrel{ \mathcal{R} }$
  (\autoref{app:cons-cdd}).
  %
  By this fact and the inductive hypothesis, $\corr{ \mathcal{R}}$ is
  a correspondence from $\copyrel{ \mathcal{R} }( \mathcal{S}',
  \mathcal{C}, R)$ to $\mathcal{S}$.

  % final claim: expand returns an expansion:
  $\expandaux{ \mathcal{R} }$ returns its parameter at some step, by
  \autoref{alg:expand}.
  %
  Therefore, $\expandaux$ returns an expansion of $\mathcal{S}$.

  % prove that returned system is CDD:
  For $\mathcal{S}' \in \chcs{ \ctxs{ \mathcal{R} } }$ and $\eta :
  \ctxs{\mathcal{R}} \to \mathcal{R}$, if $(\mathcal{S}', \eta) =
  \expand{ \mathcal{R} }( \mathcal{S} )$, then $\sharingclause(
  \mathcal{S}') = \none$, by the definition of $\expand{ \mathcal{R}
  }$.
  %
  If $\sharingclause(\mathcal{S}') = \none$, then $\mathcal{S}'$ is
  CDD, by the definition of $\sharingclause$ and CDD systems
  (\autoref{defn:cdds}).
  %
  Therefore, $\mathcal{S}'$ is CDD.
\end{proof}
%
Furthermore, $\expand{ \mathcal{R} }$ returns a \emph{minimal} CDD
expansion of its input.
%
This fact is not required to prove \autoref{thm:corr}, and thus a
complete proof is withheld.

% lem: VC's characterize system sat:
For each CHC system $\mathcal{S}$, $\mathcal{S}$ has a solution if and
only if the verification condition of $\mathcal{S}$ is unsatisfiable.
%
\begin{lem}
  \label{lem:vc}
  For $\mathcal{R} \in \relpreds$ and $\mathcal{S} \in \chcs{
    \mathcal{R} }$ that is CDD and solvable, $\vc{ \mathcal{R} }(
  \mathcal{S} )$ is unsatisfiable.
\end{lem}
%
\begin{proof}
  %
  Assume that $\mathcal{S}$ has solution $\sigma$ and $m$ is a model
  of $\vc{ \mathcal{R} }( \mathcal{S})$.
  %
  There are some relational predicates $\mathcal{R}' \subseteq
  \mathcal{R}$ containing $\queryof{ \mathcal{S} }$ such that for each
  $R' \in \mathcal{R}'$, $m(\used{R})$ holds, by the definition of
  $\vc{ \mathcal{R} }$ (\autoref{sec:solve-cdd}).
  %
  For each $R'$, the interpretation of $m$ over $\varsof{ \mathcal{R}'
  }$ must satisfy $\sigma(R')$, by well-founded induction on $\deps_S$
  restricted to $\mathcal{R}'$.
  %
  But $\sigma(\queryof{ \mathcal{S} }) = \false$, by the definition of
  a solution of a CHC system (\autoref{defn:chc-soln}).
  %
  Therefore, there can be no such model $m$.
  %
  Therefore, $\vc{ \mathcal{R} }( \mathcal{S} )$ is unsatisfiable.
\end{proof}

\begin{lem}
  \label{lem:solve-aux}
  For $\mathcal{R} \in \relpreds$ and $\mathcal{S} \in \chcs{
    \mathcal{R} }$ such that $\vc{ \mathcal{R} }(\mathcal{S})$ is
  unsatisfiable, $\solvecdd{ \mathcal{R} }(\mathcal{S})$ is a solution
  of $\mathcal{S}$.
\end{lem}
%
\begin{proof}
  % overview of structure:
  Proof by induction on the evaluation of $\solveaux$ on
  interpretation $\sigma$.
  %
  The inductive fact is that at each step of evaluation, $\sigma$ is a
  partial solution of $\mathcal{S}$ such that $\sigma(\mathcal{S})$
  % base case:
  For the base case, \solveaux is initially called on the empty
  interpretation.
  %
  Therefore, $\vc{\mathcal{R}}(\sigma(\mathcal{S})) = \vc{
    \mathcal{R}}(\mathcal{S})$, which is unsatisfiable.

  % inductive case:
  For the inductive case, in each step, \solveaux calls itself on an
  interpretation $\sigma'$, which is its argument $\sigma$, extended
  to map relational predicate $R$ to an interpolant of $\prectr{
    \mathcal{R} }( \mathcal{S}, \sigma, R)$ and $\postctr{ \mathcal{R}
  }( \mathcal{S}, \sigma, R)$.
  %
  $\sigma'$ is a partial solution of $\mathcal{S}$ by a combination of
  the inductive hypothesis that $\sigma$ is a partial solution, the
  fact that any extension of $\sigma$ that maps $R$ to a formula
  implied by $\prectr{ \mathcal{R} }( \mathcal{S}, \sigma, R)$ is a
  partial solution, and the definition of an interpolant
  (\autoref{defn:itps}).
  %
  $\vc{ \mathcal{R} }(\sigma'(\mathcal{S}))$ is unsatisfiable by a
  combination of the inductive hypothesis that $\vc{ \mathcal{R}
  }(\sigma(\mathcal{ S }))$ is unsatisfiable, that for any extension
  $\sigma''$ of $\sigma$ that maps $R$ to a formula inconsistent with
  $\postctr{ \mathcal{R} }( \mathcal{S}, \sigma, R)$, $\vc{
    \mathcal{R} }(\sigma''(\mathcal{S}))$ is unsatisfiable, and the
  definition of an interpolant (\autoref{defn:itps}).

  % prove goal:
  $\mathcal{R}'$ (defined at \autoref{alg:solve-cdd},
  \autoref{line:cons-nodeps}) is empty only if $\mathcal{R} \setminus
  \domainof{ \sigma }$ is empty, by induction on evaluation of
  \solveaux.
  %
  Therefore, \solveaux only returns its parameter $\sigma$ if $\sigma$
  is in fact a complete interpretation.
  %
  This fact, combined with the inductive fact, imply that \solveaux
  returns a solution of $\mathcal{S}$.
\end{proof}

% SolveCDD: establish soundness
For $\mathcal{R} \in \relpreds$, $\solvecdd{ \mathcal{R} }$ is a
correct CHC solver for CDD systems over $\mathcal{R}$.
%
\begin{lem}
  \label{lem:cdd-soln-sound}
  For $\mathcal{R} \in \relpreds$, %
  $\mathcal{S} \in \chcs{ \mathcal{R} }$ such that $\mathcal{S}$ is
  CDD, and %
  $\sigma \in \interps{ \mathcal{R} }$ such that %
  $\sigma = \solvecdd{ \mathcal{R} }(\mathcal{S})$, %
  $\sigma$ is a solution of $\mathcal{S}$.
\end{lem}
%
\begin{proof}
  %
  $\vc{ \mathcal{R} }(\mathcal{S})$ is unsatisfiable, by the fact that
  $\sigma \in \interps{ \mathcal{R} }$ \autoref{alg:solve-cdd},
  \autoref{line:check-sat}.
  %
  Therefore, $\solvecdd{ \mathcal{R} }(\mathcal{S})$ is the result of
  calling \solveaux, by \autoref{alg:solve-cdd},
  \autoref{line:check-sat}.
  %
  Therefore, \autoref{lem:solve-aux} and the fact that $\vc{
    \mathcal{R} }(\mathcal{S})$ is unsatisfiable imply that \solveaux
  returns a complete solution of $\mathcal{S}$.
\end{proof}

% SolveCDD: establish completeness
\begin{lem}
  \label{lem:cdd-soln-complete}
  For $\mathcal{R} \in \relpreds$ and %
  $\mathcal{S} \in \chcs{ \mathcal{R} }$ such that $\mathcal{S}$ is
  solvable and CDD, %
  there is some $\sigma \in \interps{ \mathcal{R} }$ such that %
  $\sigma = \solvecdd{ \mathcal{R} }(\mathcal{S})$.
\end{lem}
%
\begin{proof}
  % final claim:
  $\vc{ \mathcal{R} }(\mathcal{S})$ is not satisfiable, by
  \autoref{lem:vc} and the fact that $\mathcal{S}$ is solvable.
  %
  Therefore, $\solvecdd{ \mathcal{R} }$ is the result of running
  \solveaux, by \autoref{alg:solve-cdd}.
  %
  Therefore, by \autoref{lem:solve-aux} and the fact that
  $\mathcal{S}$ is solvable, $\solvecdd{ \mathcal{R} }$ returns a
  solution of $\mathcal{S}$.
\end{proof}

% proof of main theorem:
The following is a proof of correctness of \sys
(\autoref{sec:approach}, \autoref{thm:corr}).
%
\begin{proof}
  % if returns a map, then S is a solution
  Assume that for $\sigma \in \interps{ \mathcal{R} }$, $\sigma =
  \shara{ \mathcal{R} }( \mathcal{S} ) \in \interps{ \mathcal{R} }$.
  %
  Let $\mathcal{S'} \in \chcs{ \ctxs{ \mathcal{R} } }$ and $\eta :
  \ctxs{ \mathcal{R} } \to \mathcal{R}$ be such that $(\mathcal{S}',
  \eta) = \expand{ \mathcal{R} }( \mathcal{S} )$;
  %
  $\eta$ is a correspondence from $\mathcal{S'}$ to $\mathcal{S}$, by
  \autoref{lem:expand-corr}.
  %
  There is some $\sigma' \in \interps{ \mathcal{R}' }$ such that
  $\sigma' = \solvecdd{ \ctxs{ \mathcal{R} } }( \mathcal{S}' )$, by
  the definition of $\shara{ \mathcal{R} }$.
  %
  $\sigma'$ is a solution of $\mathcal{S}'$, by
  \autoref{lem:cdd-soln-sound}.
  %
  $\collapse{ \eta }{ \sigma' }$ is a solution of $\mathcal{S}$, by
  \autoref{lem:expansion-sound}.

  % if S is solvable, then shara returns a solution
  Assume that $\mathcal{S}$ is solvable.
  %
  Let $\mathcal{S'} \in \chcs{ \ctxs{ \mathcal{R} } }$ and $\eta :
  \ctxs{ \mathcal{R} } \to \mathcal{R}$ be such that $(\mathcal{S}',
  \eta) = \expand{ \mathcal{R} }( \mathcal{S} )$.
  %
  $\mathcal{S}'$ is a CDD expansion of $\mathcal{S}$, by
  \autoref{lem:expand-corr}.
  %
  $\mathcal{S}'$ is solvable, by \autoref{lem:expansion-complete}.
  %
  $\solvecdd{ \ctxs{ \mathcal{R} } }( \mathcal{S'}) \in \interps{
    \mathcal{R'} } $, by \autoref{lem:cdd-soln-complete}.
  %
  $\shara{\mathcal{R}} \in \interps{ \mathcal{R} }$, by
  \autoref{alg:shara}.
\end{proof}

%%% Local Variables: 
%%% mode: latex
%%% TeX-master: "p"
%%% End: 
