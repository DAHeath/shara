\section{Overview}
\label{sec:overview}

%first introduce the question
In \autoref{sec:running-ex}, we describe a recursion-free CHC system,
$\mcchc$, that formulates verifying the safety of a small program,
\cc{dblAbs} (\autoref{fig:multicall-code}).
%
In \autoref{sec:solve-ex}, we show that $\mcchc$ is a
\emph{Clause-Dependent Disjoint (CDD)} system and how \sys can solve it by
encoding it into binary interpolants.
%
In \autoref{sec:not-in}, we illustrate that $\mcchc$ is not in
directly solvable classes introduced in previous work.
%

% introduce running example program
\subsection{Verifying \cc{dblAbs}: an example hierarchical program}
\label{sec:running-ex}

% include code of running example and some figure:
\begin{figure}[t]
  \centering
  \begin{floatrow}[2]
    \ffigbox[.26\textwidth] %
    { \caption{\cc{dblAbs}: an example hierarchical program.} %
      \label{fig:multicall-code} }
    { \input{code/dblAbs.java} }
    %
    \ffigbox[.7\textwidth] %
    {\caption{$\mcchc$ shown as a directed hypergraph.
        % 
        Each relational predicate is depicted as a graph node
        while each clause is represented by a hyperedge.
        Each hyperedge is labelled by the fact in the corresponding CHC\@.
        Each node has a valid corresponding interpretation,
        written in braces.
      } %
      \label{fig:ex-graph} }
    % { \includegraphics[width=\linewidth]{fig/ex-deps.pdf} }
      {\scalebox{0.75}{%
\begin{tikzpicture}
\begin{scope}[every node/.style={circle,thick,draw}]
  \node[label={0:$\{true\}$}] (L4) at (0,0) {$\cc{L}_4$};
  \coordinate[above = of L4] (L4');
  \node[label={180:$\{\cc{n} \geq 0\}$}, below left = of L4] (L6) {$\cc{L}_6$};
  \node[label={0:$\{\cc{n} < 0\}$}, below right = of L4] (L8) {$\cc{L}_8$};
  \node[label={180:$\{\cc{abs'} \geq 0\}$}, below right = of L6] (L9) {$\cc{L}_9$};
  \node[label={0:$\{\cc{res} \geq 0\}$}, below right = 1.6 and 1.2 of L9] (main) {$\cc{main}$};
  \node[label={0:$\{\cc{d} = 2*\cc{x}\}$}, above right = 1.6 and 1.2 of main] (dbl) {$\cc{dbl}$};
  \coordinate[above = of dbl] (dbl');
  \node[label={0:$\{false\}$}, below = of main, style = ultra thick] (bot) {$\bot$};
  \coordinate[above = 0.5 of main] (main');
\end{scope}

\begin{scope}[every edge/.style={draw=black, very thick}]
  \path[->, style=right] (L4') edge node {$\cc{abs} = 0$} (L4);
  \path[->, style=above left] (L4) edge node {$\cc{n} \geq 0$} (L6);
  \path[->, style=above right] (L4) edge node {$\cc{n} < 0$} (L8);
  \path[->, style=left] (L6) edge node {$\cc{abs'} = \cc{n}$} (L9);
  \path[->, style=right] (L8) edge node {$\cc{abs'} = -\cc{n}$} (L9);
  \path[-] (L9) edge[out=330, in=90] (main');
  \path[->, style=right] (dbl') edge node {$\cc{d} = 2*\cc{x}$} (dbl);
  \path[-] (dbl) edge[out=210, in=90] (main');
  \path[->, style=above right] (main') edge node {$\cc{abs'} = \cc{x} \land \cc{res} = \cc{d}$} (main);
  \path[->, style=right] (main) edge node {$\cc{res} < 0$} (bot);
\end{scope}
\end{tikzpicture}
}
}
  \end{floatrow}
\end{figure}
% walk through code of running example:
%
\cc{dblAbs} is a small procedure that doubles the absolute value of
its input and stores the result in \cc{res}.
% state the safety problem:
The program also asserts that \cc{res} is greater than or equal to $0$
before exiting.
% talk about solving Horn Clauses:
Verifying this assertion reduces to solving a recursion-free
CHC system over a set of uninterpreted predicates that represent
the control locations in \cc{dblAbs}.
%
In particular, one such system $\mcchc$ is
%
\begin{align}
\label{chc}
  % semantic constraint of dbl:
\cc{dbl}(\cc{x}, \cc{d}) &\gets \cc{d} = 2 * \cc{x} \\
  % semantic constraint of line 3:
  \cc{L}_4(\cc{n}, \cc{abs}) &\gets  \cc{abs} = 0 \\
  % semantic constraint of line 5 (then branch)
  \cc{L}_6(\cc{n}, \cc{abs}) &\gets \cc{L}_4(\cc{n}, \cc{abs}) \land \cc{n} \ge 0 \\
  % semantic constraint of line 6 (else branch)
  \cc{L}_8(\cc{n}, \cc{abs}) &\gets \cc{L}_4(\cc{n}, \cc{abs}) \land \cc{n} < 0 \\
  % assign branch
  \cc{L}_9(\cc{n}, \cc{abs'}) &\gets \cc{L}_6(\cc{n},\cc{abs}) \land \cc{abs'} = \cc{n} \\
   \cc{L}_9(\cc{n}, \cc{abs'}) &\gets \cc{L}_8(\cc{n},\cc{abs}) \land \cc{abs'} = -\cc{n} \\
  % semantic constraint of main procedure:
   \cc{main}(\cc{n},\cc{res}) &\gets \cc{L}_9(\cc{n}, \cc{abs'}) \land
                                     \cc{dbl}(\cc{x}, \cc{d})
    \land \cc{abs'} = \cc{x} \land \cc{res} = \cc{d} \\
  %
    \bot &\gets \cc{main}(\cc{n}, \cc{res}) \land \cc{res} < 0 
\end{align}
%
While $\mcchc$ has been presented as the result
of a translation from hierarchical program \cc{dblAbs}, \sys is
purely a solver for CHC systems: it does not require access to the
concrete representation of a program, or for a given CHC system to be
the result of translation from a program at all.

% solving the running example
\subsection{$\mcchc$ as a Clause-Dependent Disjoint System}
\label{sec:solve-ex}
% state it is CDD
The recursion-free CHC system $\mcchc$ is a
\emph{Clause-Dependent-Disjoint} (CDD) system.
% state the defn of CDD:
A CDD system is a recursion-free CHC system such that
the dependencies between predicates in the system obey certain rules.
%
One predicate is directly dependent on another if the second appears
in the body of a clause where the first is the head.
%
Based on this, we say that a clause is dependence disjoint if
\textbf{(1)} no two predicates in the body share any transitive
dependencies on other predicates and \textbf{(2)} no two predicates in
the body are the same.
% give a example
As an example, clause $(7)$ is dependence disjoint. Clause $(7)$ uses
both $\cc{L}_9$ and $\cc{dbl}$ in its body.
%
The transitive dependency of $\cc{L}_9$ is the set
$\{\cc{L}_4,\cc{L}_6,\cc{L}_8\}$, while the transitive dependency of
\cc{dbl} is the empty set.
%
Therefore, their transitive dependencies are disjoint:
$\{\cc{L}_4,\cc{L}_6,\cc{L}_8\} \cap \varnothing = \varnothing$.
%
All other clauses in $\mcchc$ have at most one uninterpreted predicate
in the body, so they are trivially disjoint dependent.
%
When each clause in a CHC system is dependence disjoint, it is a CDD
system. Therefore $\mcchc$ is a CDD system.
%
The fact that $\mcchc$ is CDD corresponds to the fact that there no
statements executed more than once within one execution path.
\DAH{elaborate}
%
The formal definition of CDD and its key properties are given in
\autoref{sec:CDD-defn}.
The formal definition of transitive dependency is given in\QZ{reference}.
%

% restate defn of CDD systems:
\sys solves CDD systems directly by issuing a binary interpolation
query for each uninterpreted predicate in topological order.
%
Each uninterpreted predicate $P$ has a valid interpretation which can
be computed by interpolating
\textbf{(1)} the \emph{pre}-formula, constructed from the solutions of
each uninterpreted predicate $P$ directly depends on and constraints
in clauses where $P$ is the head and
\textbf{(2)} the \emph{post}-formula, constructed from
all clauses where the head transitively depends on $P$.
%

We will walk through the construction of an interpolation query for
$\cc{L}_9$ by hand.  By the time \sys reaches $\cc{L}_9$ it will
already have solutions for $\cc{L}_4$, $\cc{L}_6$, $\cc{L}_8$.
Possible interpretations of these predicates are shown in
\autoref{fig:ex-graph}.
%
A solution for \cc{L9} can be constructed by interpolating \textbf{(1)}
the \emph{pre}-formula, constructed as solutions generated for \cc{L4}, \cc{L6}
the constraints in clauses (5) and (6) and boolean variables%
\textbf{(2)} the \emph{post}-formula constructed from constraints in
$\mcchc$ that all constraints after \cc{L9}, constraints for \cc{dbl}
and boolean variables.
%
In particular, the \emph{pre}-formula for \cc{L9} is 
\begin{gather}
  ((b_{L_8}\land \cc{abs'}= -\cc{n})
    \lor (b_{L_6} \land \cc{abs}=\cc{n}))
  \land (\neg b_{L_6} \lor \cc{n} \ge 0)
    \land (\neg b_{L_8} \lor \cc{n}<0)
\end{gather}
%
, $b_{L8}$ and $b_{L6}$ are boolean variables corresponding to predicates \cc{L8} and \cc{L6}.
%
The \emph{post}-formula for \cc{L9} is
\begin{gather}
  (b_{main} \land \cc{res'} < 0 )
    \land (\neg b_{main} \lor (b_{dbl} \land \cc{abs'}= \cc{x} \land \cc{res}=\cc{d} ))
  \land (\neg b_{dbl} \lor \cc{d} = 2*\cc{x})
\end{gather}
, $b_{main}$ and $b_{dbl}$ are boolean variables corresponding to
predicates \cc{main} and \cc{dbl}.
%
A detailed and formal description for the procedure of solving a CDD system is given in
\autoref{sec:solve-cdd}.


\subsection{$\mcchc$ is not in other system}
\label{sec:not-in}
%
There are three classes of directly solvable systems has been introduced
in previous work(as described in \autoref{sec:intro}).
%
However, $\mcchc$ does not belong to any of these three classes.
%

% tree solver 
In particular, previous work has introduced tree structure systems\cite{bjorner13,heizmann10},
which each uninterpreted predicate in the system is the head of at most one clause.
%
Such systems do not include $\mcchc$, because predicate \cc{L9} is the head 
of clause (5) and (6).
%
CHC solvers based on tree structure system has to reduce solving $\mcchc$ to
solving two tree strucutre systems that contains by $\mcchc$, that one corresponds
to the tree branch and the other corresponds to the false branch.
% not that bad 
There are works has been proposed which in some cases enumerating 
and solving all derivation trees 
of a given recursion-free
system~\cite{mcmillan14}.
%
However, such solvers may enumerate all derivations tree of a given system
in the worst case.

% Body-Disjoint:
Previous work has introduced \emph{body-disjoint}
systems~\cite{rummer13a,rummer13b}, 
which each uninterpreted predicate shows in the body of at most one
clause, and shows in the body of such a clause at most once;
%
However, such systems cannot model a program with multiple
control paths that share a common subpath, typically modeled as a CHC
system with a uninterpreted predicate that occurs in the body of multiple clauses.
%
$\mcchc$ is not a body-disjoint system because the uninterpreted
predicates \cc{L4} shows in the body of both clauses (3) and (4).
%
Such a solver, given $\mcchc$, generates a body-disjoint system
consisting of two copies of the sub-systems with head relational
predicates \cc{L4}.

%linear systems:
Previous work also introduced linear system~\cite{albarghouthi12a}, 
which each clause's body has at most one uninterpreted predicate.
%
However, such systems cannot directly model the control flow of a
program that contains procedure calls.
%
$\mcchc$ is not a linear system because the body of clause (7) has two
predicates \cc{L9} and \cc{dbl}.
%
CHC solvers that solve linear systems, $\mcchc$, effectively inline constraints on
relational predicates that occur in non-linear clauses~\cite{albarghouthi12b}.
%
Such approaches can, in general, generate an inlined system that is
exponentially larger than a given system.

%% talk about expansion:
The class of CDD systems strictly contains these three classes, 
but does not contain all
recursion-free systems.
%
E.g., a recursion-free system that models \cc{dblAbs} 
call \cc{dbl} twice on \cc{abs} in sequence would not
be CDD.
%
\sys, given an arbitrary recursion-free CHC system $S$,
solves it by generating an equivalent CDD system $S'$ and
solving $S'$ directly.
%
In general, $S'$ may have size exponential in the size of
$S$.
%
In practice, a suitable equivalent system can be generated using
heuristics analogous to those used to generate compact verification
conditions of hierarchical programs~\cite{flanagan01,lal-qadeer15}.
%
A general procedure for constructing a CDD expansion of a given CHC
system is given in \autoref{app:cons-cdd}.

%%% Local Variables: 
%%% mode: latex
%%% TeX-master: "p"
%%% End: 
