\section{Overview}
\label{sec:overview}

%first introduce the question
In \autoref{sec:running-ex}, we describe a recursion-free CHC system,
$\mcchc$ (\autoref{fig:chc}), that models the safety of a small program,
\cc{dblAbs} (\autoref{fig:multicall-code}).
%
In \autoref{sec:solve-ex}, we show that $\mcchc$ is a CDD system and
how \sys can solve it by encoding it into binary interpolants.
%
In \autoref{sec:not-in}, we illustrate that $\mcchc$ is not in
directly solvable classes introduced in previous work.
%

% introduce running example program
\subsection{Verifying \cc{dblAbs}: an example hierarchical program}
\label{sec:running-ex}

% include code of running example and some figure:
\begin{figure}[t]
  \centering
  \begin{floatrow}[2]
    \ffigbox[.3\textwidth] {%
      \caption{\cc{dblAbs}: an example hierarchical program.} %
      \label{fig:multicall-code} }
    {\lstinputlisting[
      basicstyle=\small,
      xleftmargin=\parindent,
      numbers=left,
      morekeywords={def, return, if, else, assert}]{code/dblAbs.java}
        % \input{code/dblAbs.java}}
    }
    %
    \ffigbox[.66\textwidth]{%
      \caption{A CHC system that models the safety condition of $\cc{dblAbs}$,
    named $\mcchc$.}}{%
      \label{fig:chc}
      \begin{align}
        % semantic constraint of dbl:
        \cc{dbl}(\cc{x}, \cc{d}) &\gets \cc{d} = 2 * \cc{x} \\
        % semantic constraint of line 3:
        \cc{L}_4(\cc{n}, \cc{abs}) &\gets  \cc{abs} = 0 \\
        % semantic constraint of line 5 (then branch)
        \cc{L}_6(\cc{n}, \cc{abs}) &\gets \cc{L}_4(\cc{n}, \cc{abs}) \land \cc{n} \ge 0 \\
        % semantic constraint of line 6 (else branch)
        \cc{L}_8(\cc{n}, \cc{abs}) &\gets \cc{L}_4(\cc{n}, \cc{abs}) \land \cc{n} < 0 \\
        % assign branch
        \cc{L}_9(\cc{n}, \cc{abs'}) &\gets \cc{L}_6(\cc{n},\cc{abs}) \land \cc{abs'} = \cc{n} \\
        \cc{L}_9(\cc{n}, \cc{abs'}) &\gets \cc{L}_8(\cc{n},\cc{abs}) \land \cc{abs'} = -\cc{n} \\
        % semantic constraint of main procedure:
        \cc{main}(\cc{n},\cc{res}) &\gets \cc{L}_9(\cc{n}, \cc{abs'}) \land
        \cc{dbl}(\cc{x}, \cc{d})
        \land \cc{abs'} = \cc{x} \land \cc{res} = \cc{d} \\
        %
        False &\gets \cc{main}(\cc{n}, \cc{res}) \land \cc{res} < 0 
      \end{align}
    }
  \end{floatrow}
\end{figure}
% walk through code of running example:
%
\cc{dblAbs} is a small procedure that doubles the absolute value of
its input and stores the result in \cc{res}.
% state the safety problem:
The program also asserts that \cc{res} is greater than or equal to $0$
before exiting.
% talk about solving Horn Clauses:
Verifying this assertion reduces to solving a recursion-free
CHC system over a set of uninterpreted predicates that represent
the control locations in \cc{dblAbs}.
%
In particular, one such system, $\mcchc$, is shown in
\autoref{fig:chc}.
%
%
While $\mcchc$ has been presented as the result
of a translation from \cc{dblAbs}, \sys is
purely a solver for CHC systems: it does not require access to the
concrete representation of a program, or for a given CHC system to be
the result of translation from a program at all.

% solving the running example
\subsection{$\mcchc$ as a Clause-Dependence Disjoint System}
\label{sec:solve-ex}
% state it is CDD
The recursion-free CHC system $\mcchc$ is a
\emph{Clause-Dependence Disjoint} (CDD) system.
% state the defn of CDD:
A CHC system can be classified as CDD when each clause satisfies the
following rules:
%
\textbf{(1)} no two predicates in the body of the same clause share
any transitive dependencies on other predicates and
\textbf{(2)} no clause has more than one occurrence of a given
predicate in the body.
% give a example
As an example, clause $(7)$ is dependence disjoint. 
%
Two predicates $\cc{L}_9$ and $\cc{dbl}$ are in its body.
%
The transitive dependency of $\cc{L}_9$ is the set
$\{\cc{L}_4,\cc{L}_6,\cc{L}_8\}$ while the transitive dependency of
\cc{dbl} is the empty set.
%
Therefore, their transitive dependencies are disjoint:
$\{\cc{L}_4,\cc{L}_6,\cc{L}_8\} \cap \varnothing = \varnothing$.
%
All other clauses in $\mcchc$ have at most one uninterpreted predicate
in the body, so they are trivially disjoint dependent.
%
Therefore $\mcchc$ is a CDD system.
%
%
The formal definition of CDD and its key properties are given in
\autoref{sec:CDD-defn}.
The formal definition of transitive dependency is given in
\autoref{sec:chcs}.
%

% restate defn of CDD systems:
\sys solves CDD systems directly by issuing a binary interpolation
query for each uninterpreted predicate in topological order.
%
Each interpretation of a predicate $P$ can be computed by
interpolating
\textbf{(1)} the \emph{pre}-formula, constructed from clauses where
$P$ is the head, and
\textbf{(2)} the \emph{post}-formula, constructed from
all clauses where the head transitively depends on $P$.
%

For example, consider $\cc{L}_9$. By the time \sys attempts to
synthesize an interpretation for $\cc{L}_9$ it will have solutions for
$\cc{L}_4$, $\cc{L}_6$, $\cc{L}_8$.
%
Possible interpretations of these predicates are shown in
\autoref{fig:ex-graph}.
%
The pre-formula is constructed from the bodies of clauses where
$\cc{L}_9$ is the head.
%
Each relational predicate, $P$,
is replaced by a corresponding boolean indicator variable, $\cc{b}_P$.
%
Each boolean indicator variable implies the solution for its
predicate,
encoded as the disjunction of the negation of the boolean
indicator variable and the solution.
%
In particular, the pre-formula for $\cc{L}_9$ is constructed from
clauses (5) and (6):
\begin{gather}
  ((\cc{b}_{\cc{L}_6} \land \cc{abs'} = \cc{n})
  \lor
  (\cc{b}_{\cc{L}_8} \land \cc{abs'} = -\cc{n}))
  \land
  (\neg \cc{b}_{\cc{L}_6} \lor \cc{n} \ge 0)
  \land
  (\neg \cc{b}_{\cc{L}_8} \lor \cc{n} < 0)
\end{gather}
%
The post-formula is constructed from clauses that transitively depend
on $\cc{L}_9$. Again, we replace relational predicates by
corresponding boolean indicators. However, we omit the boolean indicator for
$\cc{L}_9$. The post-formula is composed from clauses (1), (7), and
(8):
\begin{gather}
  (\neg \cc{b}_{\cc{dbl}} \lor \cc{d} = 2*\cc{x})
  \land
  (\neg \cc{b}_{\cc{main}} \lor
    (\cc{b}_{\cc{dbl}}
    \land \cc{abs'}= \cc{x}
    \land \cc{res}=\cc{d} ))
  \land
  (\cc{b}_{\cc{main}} \land \cc{res'} < 0)
\end{gather}
%
Interpolating the pre and post formulas yields an interpretation
of $\cc{L}_9$: $\cc{abs'} \geq 0$.
%
The procedure for solving a CDD system is described in formal detail
in \autoref{sec:solve-cdd}.

\subsection{$\mcchc$ is not in other recursion-free classes}
\label{sec:not-in}
%
In this section, we show that $\mcchc$ is not in other
known classes of recursion-free CHC systems. 
%
Specifically, we will
discuss body-disjoint systems and linear systems.
%

% tree solver
Body-disjoint (or derivation tree)
systems~\cite{mcmillan14,bjorner13,heizmann10,rummer13a,rummer13b} are
a class of recursion-free CHC system where each uninterpreted
predicate appears in the body of at most one clause and appears in
such a clause exactly once.
%
Such systems cannot model a program with multiple control paths that
share a common pre-subpath, typically modeled as a CHC system with a
uninterpreted predicate that occurs in the body of multiple clauses.
%
$\mcchc$ is not a body-disjoint system because $\cc{L}_4$ shows in the
body of both clause (3) and clause (4).
%
In order to solve for $\mcchc$, a solver that uses
body-disjoint systems would have to create a fresh copy of
$\cc{L}_4$. 
%
Worse, if $\cc{L}_4$ had dependencies, each would also need to be copied.

%linear systems:
Previous work has also introduced the class of linear
systems~\cite{albarghouthi12a}, where the body of each clause has at
most one uninterpreted predicate.
%
However, such systems cannot directly model the control flow of a
program that contains procedure calls.
%
$\mcchc$ is not a linear system because the body of clause (7) has two
predicates \cc{L9} and \cc{dbl}.
%
CHC solvers that use linear systems effectively inline the
constraints for relational predicates that occur in non-linear
clauses~\cite{albarghouthi12b}.
%
In the case of $\mcchc$, inlining the constraints of $\cc{dbl}$ is
efficient, but in general such approaches can generate systems that
are exponentially larger than the input. For example, if the procedure
$\cc{dbl}$ were called more than once in $\cc{dblAbs}$ then multiple
copies of the body of the procedure would be inlined. And if the body
of this procedure were large, the inlining could become
prohibitively expensive.

%% talk about expansion:
\begin{figure}[t]
  \centering
  \begin{floatrow}[2]
    \ffigbox[.45\textwidth] {%
      {\caption{$\mcchc$ as a directed hypergraph. Each relational
          predicate is depicted as a graph node while each clause is
          represented by a hyperedge. Each hyperedge is labeled by the
          fact in the corresponding CHC\@. Each node has a valid
          corresponding interpretation, written in braces.
      }\label{fig:ex-graph}}}
      {\scalebox{0.75}{%
\begin{tikzpicture}
\begin{scope}[every node/.style={circle,thick,draw}]
  \node[label={0:$\{true\}$}] (L4) at (0,0) {$\cc{L}_4$};
  \coordinate[above = of L4] (L4');
  \node[label={180:$\{\cc{n} \geq 0\}$}, below left = of L4] (L6) {$\cc{L}_6$};
  \node[label={0:$\{\cc{n} < 0\}$}, below right = of L4] (L8) {$\cc{L}_8$};
  \node[label={180:$\{\cc{abs'} \geq 0\}$}, below right = of L6] (L9) {$\cc{L}_9$};
  \node[label={0:$\{\cc{res} \geq 0\}$}, below right = 1.6 and 1.2 of L9] (main) {$\cc{main}$};
  \node[label={0:$\{\cc{d} = 2*\cc{x}\}$}, above right = 1.6 and 1.2 of main] (dbl) {$\cc{dbl}$};
  \coordinate[above = of dbl] (dbl');
  \node[label={0:$\{false\}$}, below = of main, style = ultra thick] (bot) {$\bot$};
  \coordinate[above = 0.5 of main] (main');
\end{scope}

\begin{scope}[every edge/.style={draw=black, very thick}]
  \path[->, style=right] (L4') edge node {$\cc{abs} = 0$} (L4);
  \path[->, style=above left] (L4) edge node {$\cc{n} \geq 0$} (L6);
  \path[->, style=above right] (L4) edge node {$\cc{n} < 0$} (L8);
  \path[->, style=left] (L6) edge node {$\cc{abs'} = \cc{n}$} (L9);
  \path[->, style=right] (L8) edge node {$\cc{abs'} = -\cc{n}$} (L9);
  \path[-] (L9) edge[out=330, in=90] (main');
  \path[->, style=right] (dbl') edge node {$\cc{d} = 2*\cc{x}$} (dbl);
  \path[-] (dbl) edge[out=210, in=90] (main');
  \path[->, style=above right] (main') edge node {$\cc{abs'} = \cc{x} \land \cc{res} = \cc{d}$} (main);
  \path[->, style=right] (main) edge node {$\cc{res} < 0$} (bot);
\end{scope}
\end{tikzpicture}
}
}
    \ffigbox[.45\textwidth]
      {\caption{The hierarchy of classes of recursion-free CHC
          systems. Body Disjoint and Linear systems are subsumed by
          CDD systems. Solving Directly Solvable CHC systems is in
          co-NP while solving general, recursion-free systems is
          co-NEXPTIME Complete.
      }\label{fig:hierarchy}}
      {\begin{tikzpicture}
\begin{scope}
  \node at (0, 0) (rec) {%
    \begin{tabular}{c}
      Recursion Free\\\textsc{co-Nexptime Complete}
    \end{tabular}
  };

  \node[below = 0.5 of rec] (dir) {%
    \begin{tabular}{c}
      Directly Solvable\\\textsc{co-NP}
    \end{tabular}
  };

  \node[draw, rounded corners=0.5cm, below = 0.5 of dir] (cdd) {%
    \begin{tabular}{c}
      Clause-Dependence Disjoint\\\textsc{co-NP}
    \end{tabular}
  };

  \node[below left = 0.5 and -1.1 of cdd] (lin) {%
    \begin{tabular}{c}
      Linear\\\textsc{co-NP}
    \end{tabular}
  };

  \node[below right = 0.5 and -1.1 of cdd] (bd) {%
    \begin{tabular}{c}
      Body-Disjoint\\\textsc{co-NP}
    \end{tabular}
  };

\end{scope}
\begin{scope}[every edge/.style={draw=black, very thick}]
  \path[-] (bd) edge (cdd);
  \path[-] (lin) edge (cdd);
  \path[-] (cdd) edge (dir);
  \path[-] (dir) edge (rec);
\end{scope}
\end{tikzpicture}
}
  \end{floatrow}
\end{figure}

The hierarchy of discussed classes of recursion-free systems is
depicted in \autoref{fig:hierarchy}.
%
As shown, the class of CDD systems is a superclass of both the class
of body-disjoint systems and the class of linear systems. So any
recursion-free system that is efficiently expressible in body-disjoint
or linear form is also efficiently expressible in CDD form.
%
In addition, some systems which are expensive to express in
body-disjoint or linear form are efficiently expressible in CDD form.
%
\sys~takes advantage of this fact when solving input systems.
Given an arbitrary recursion-free CHC system $S$, \sys~reduces $S$ to
a CDD system $S'$ and solves $S'$ directly.
%
In general, $S'$ may have size exponential in the size of
$S$.
%
However, \sys~generates CDD systems via heuristics analogous to
those used to generate compact verification conditions of hierarchical
programs~\cite{flanagan01,lal-qadeer15}.
%
In practice these heuristics often yield CDD systems which are
small with respect to the input system.
%
A general procedure for constructing a CDD expansion of a given CHC
system is given in \autoref{app:cons-cdd}.
