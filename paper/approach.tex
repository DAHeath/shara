% walk through the approach in technical detail
\section{Technical Approach}
\label{sec:approach}
%
This section presents the technical details of our approach.
%
\autoref{sec:CDD-defn} presents the class of Clause-Dependence Disjoint
systems and its key properties.
%
\autoref{sec:solve-cdd} describes how \sys solves CDD systems
directly.
%
\autoref{sec:core-solver} describes how \sys solves a given
recursion-free system by solving an CDD system.
%
Proofs of all theorems stated in this section are in appendix
\autoref{app:char} and \autoref{app:corr}.

\subsection{Clause-Dependence Disjoint Systems}
\label{sec:CDD-defn}
%
The key contribution of our work is the introduction of the class of
Clause-Dependence Disjoint (CDD) CHC systems:
%
\begin{defn}
  \label{defn:cdds}
  For a given recursion-free CHC system $\cc{S}$,
  %
  if for all sibling pairs, $P, Q \in \predof{S}$,
  %
  the transitive dependencies of $P$ and $Q$ are disjoint ($\tdepsof{P} \cap
  \tdepsof{Q} = \emptyset$)
  %
  and no predicate shows more than once in the body of a single clause,
  %
  then $\cc{S}$ is \emph{Clause-Dependence Disjoint (CDD)}.
\end{defn}
%
CDD systems model hierarchical programs with branches and procedure calls such
that each execution path invokes each statement at most once.
%
\begin{ex}
  The CHC system $\mcchc$ is a CDD system. An argument is given in
  \autoref{sec:solve-ex}.
\end{ex}

% talk about previous system
As discussed in \autoref{sec:overview}, CDD is a superclass of the union of
class of body-disjoint systems and the class of linear systems.
%
For a given recursion-free system $S$, if each uninterpreted predicate $Q \in
\predof{S}$, appears in at most one in the body of a clause and, no
predicate appears more than once in the body of a single clause,
then $S$ is \emph{body-disjoint}~\cite{rummer13a,rummer13b}.
%
If the body of each clause in $\cc{S}$ contains at most one relational
predicate, then $\cc{S}$ is \emph{linear}~\cite{albarghouthi12a}.

% talk about linear system
\begin{thm}
\label{thm:cdd-contains}
  The class of CDD systems is a strict superset of the union of the
  class of body-disjoint systems and the class of linear systems.
\end{thm}
%
Proof is given in \autoref{app:char}.

% describe how to solve a CDD system
\subsection{Solving a CDD system}
\label{sec:solve-cdd}

% algorithm for constructing a CDD system:
\begin{algorithm}[t]
  % Declare IO markers.
  \SetKwInOut{Input}{Input}
  %
  \SetKwInOut{Output}{Output}
  % Declare sub-program (procedure) markers.
  \SetKwProg{myproc}{Procedure}{}{}
  % Inputs: a heap program and an error location.
  \Input{A CDD System $\cc{S}$.}
  % Output: inductive invariants.
  \Output{If $\cc{S}$ is solvable, then a solution of
    $\cc{S}$; %
    otherwise, the value $\nosoln$.}
  % 
  \myproc{$\solvecdd(\cc{S})$ %
    \label{line:solve-begin}}{%
    $\sigma \assign \emptyset$ \\
    $\cc{Preds} \assign \topSort(\predof{S})$ \\
    \label{line:top-sort}}{%
    \For {$P \in \cc{Preds}$}
    {$\cc{interpolant} \assign \solveitp(\prectr(P,\sigma),\postctr(P,\sigma))$
      \label{line:interp}\\
     \Switch{\cc{interpolant}}{%
       \lCase{\cc{SAT}:}{\Return{$\nosoln$}}\label{line:interp-sat}
       \lCase{$\cc{I}$:}{$\sigma$ [$P$] $\assign$ $\cc{I}$}\label{line:interp-valid}
     }
    }
    \Return{$\sigma$}\label{line:solve-done}
    }
  %
  \caption{$\solvecdd$: for a CDD system $\cc{S}$, returns a
    solution to $\cc{S}$ or the value $\none$ to denote that
    $\cc{S}$ has no solution.}
  \label{alg:solve-cdd}
\end{algorithm}


% introduce solveCDD
\autoref{alg:solve-cdd} presents $\solvecdd$, a procedure designed to solve
CDD systems.
%
Given a CDD system $\cc{S}$, $\solvecdd$ topologically sorts the uninterpreted
predicates in $\cc{S}$ based on their dependency relations
(\autoref{line:top-sort}).
%
Then, the algorithm calculates interpretations for each predicate in this order
by invoking $\solveitp$ (\autoref{line:interp}).
%
$\solveitp$ computes a binary interpolant of the pre and post formulas of the
given predicate, where these formulas are based on the current, partial
solution.
%
The pre and post formulas are computed respectively by $\prectr$ and
$\postctr$, which we define in \autoref{sec:cons-pre} and \autoref{sec:cons-post}.
%
It is possible that the pre and post formulas may be mutually
satisifiable, in which case $\solveitp$ returns \cc{SAT}
(\autoref{line:interp-sat}). In this case, $\solvecdd$ returns
$\nosoln$ to indicate that $\cc{S}$ is not solvable.
%
Otherwise, $\solvecdd$ updates the partial solution by setting the
interpretation of $P$ to $\cc{I}$ (\autoref{line:interp-valid}).
%
Once all predicates have been interpolated, $\solvecdd$ returns the complete
solution, $\sigma$ (\autoref{line:solve-done}).

%
\begin{ex}
  Given the CDD system $\mcchc$, $\solvecdd$ may generate interpolation
  queries in any topological ordering of the dependency relations.
  %
  One such ordering is $\cc{L}_4$, $\cc{L}_6$, $\cc{L}_8$, $\cc{L}_9$, \cc{dbl},
  \cc{main}.
\end{ex}

% solve-cdd in coNP
\begin{thm}
  \label{thm:solve-cp}
  Given a CDD system $S$, $\solvecdd$ either returns the solution of $S$ or $\none$
  in co-NP time.
\end{thm}
Proof is given in \autoref{app:solve-cp}.

%
\subsubsection{Constructing constraints for predicates}
%
In order to construct efficiently sized pre and post formulas for
relational predicates, we need a method of compactly expressing the
constraints on a given predicate.
%
For a CDD system $S$, a predicate $P \in \predof{S}$, and a partial
solution $\sigma$ that maps predicates to their solutions, the formula
$\ctrof{P,\sigma}$ is a compact representation of the constraints of
$P$.
%
If $\sigma$ does not contain $P$, then the constraint of $P$ is
constructed from the clauses where $P$ is the head.  When $\sigma$
does contain $P$, $\ctrof{P,\sigma}$ is a lookup from $\sigma$.
%
Each $P \in \predof{S}$ has a corresponding boolean variable
$\cc{b}_P$:
%
\[
  \ctrof{P,\sigma}=
  \begin{dcases}
    \biglor_{(\cc{C}_i \in \cc{S}) \land (\headof{C_i} = P)}
    \left( \consof{C_i} \land %
      \bigland_{ \cc{Q} \in \bodyof{C_i}} \cc{b}_Q
    \right),
  &\text{if } P \notin \sigma\\
  \sigma[P], &\text{if } P \in \sigma
  \end{dcases}
\]
%
% introduce the isused predicates:
The counterexample characterization of $P$ is a small extension of the
compact constraint of $P$.
%
It states that if $P$ is used
(meaning $\cc{b}_{P} = True$), then the constraint of $P$ must hold:
\[
  \vc{P,\sigma} = \neg \cc{b}_{P} \lor \ctrof{P,\sigma}
\]


\begin{ex}
  \label{ex:ctr}
  When $\solvecdd$ solves predicate $\cc{L}_9$ in $\mcchc$, it generates a
  constraint based on clauses (5) and (6):
  $$\ctrof{\cc{L}_9,\sigma} =
    (\cc{abs'} = \cc{n} \land \cc{b}_{\cc{L}_6})
    \lor
    (\cc{abs'} = -\cc{n} \land \cc{b}_{\cc{L}_8})$$
  The counterexample characterization for $\cc{L}_9$ is based on its boolean
  indicator and its constraint:
  $$\vc{\cc{L}_9,\sigma} = \neg \cc{b}_{\cc{L}_9} \lor \ctrof{\cc{L}_9,\sigma}$$
\end{ex}

% pre-constraint:
\subsubsection{Constructing pre-formulas for predicates}
\label{sec:cons-pre}
%
$\prectr(P, \sigma)$ denotes the pre-formula for an arbitrary predicate $P$
with respect to the partial solution map, $\sigma$.
%
Due to the topological ordering, when $\solvecdd$ attempts to solve $P$, the
interpretations for all dependencies of $P$ will be stored in
$\sigma$.
%
The pre-formula is built from these interpretations together with boolean
indicators of the dependencies and the constraint of $P$:
%
\[
\prectr(P,\sigma) =
  \ctrof{P,\sigma} \land
    \left(
      \bigland_{\cc{Q} \in \depsof{P}}
      \left(\neg b_Q \lor \sigma[ \cc{Q} ]  \right)
    \right)
\]
%
\begin{ex}
  \label{ex:pre-ctr}
  When $\solvecdd$ solves predicate $\cc{L}_9$ in $\mcchc$,
  $\sigma$ maps $\cc{L}_6$ to $n \ge 0$ and $\cc{L}_8$ to $n < 0$.
  %
  The pre-formula for $\cc{L}_9$ under $\sigma$ is therefore:
  $$\ctrof{\cc{L}_9}
  \land (\neg \cc{b}_{\cc{L}_6} \lor n \ge 0) \land (\neg \cc{b}_{\cc{L}_8}
  \lor n < 0)$$
  %
  The formula $\ctrof{\cc{L}_9,\sigma}$ is given in~\autoref{ex:ctr}.
\end{ex}

\subsubsection{Constructing post-formulas for predicates}
\label{sec:cons-post}
$\postctr(P, \sigma)$ denotes the post-formula for an arbitrary predicate $P$
with respect to the partial solution map, $\sigma$. A valid
post-formula is mutually inconsistent with the solution of $P$.
%
and is constructed based on the predicates which depend on $P$.
%
% dependents of R:
Let $D_0$ be the \emph{reflexive transitive dependents} of $P$ in $S$,
% clause siblings of dependents:
let $D_1$ be the siblings in $S$ of $(D_0 \union {P})$,
%
let $D_2$ be all transitive dependencies of $D_1$, and
% collect all predicates that we define a constraint over:
let $D = D_0 \union D_1 \union D_2$.
% define instantiation of C
The post-formula for $P$ under $\sigma$ is the conjunction of counterexample
characterization of all predicates $Q \in D$ and the query clause:
%
\[
\postctr(R, \sigma) =\cc{query} \land (\bigland_{\cc{Q} \in D} \vc{Q,\sigma})
\]
% example post constraint:
\begin{ex}
  \label{ex:ctx-ctr}
  When $\solvecdd$ solves $\cc{L}_9$ in $\mcchc$, it must consider
  the dependents of $\cc{L}_9$.
  %
  The \emph{reflexive transitive dependents} $D_0$ of $\cc{L}_9$ is \{\cc{main}\}.
  %
  The siblings set $D_1$ is \{\cc{dbl}\}.
  %
  The transitive dependencies of $D_1$ is $\varnothing$. 
  %
  Therefore, the $D$ set is \{\cc{main},\cc{dbl}\}.
  %
  The \cc{query} is $\cc{b}_{main} \land res<0$.
  %
  The post-formula for $\cc{L}_9$ under $\sigma$ is:
  $$\cc{query} \land \vc{\cc{main},\sigma} \land \vc{\cc{dbl},\sigma}$$
\end{ex}

\subsection{Solving recursion-free systems using CDD systems}
\label{sec:core-solver}

% give the general procedure:
$\shara$, given a recursion-free CHC system $S$,
constructs an CDD system $S'$.
%
\sys then directly solves $S'$ and, from this solution, constructs a solution
for $S$.
% define an expansion of a CHC system:
For two given recursion-free CHC systems $S$ and $S'$, if
there is a homomorphism from $\predof{S'}$ to $\predof{S}$ that
preserves the relationship between the clauses of $S'$ in
the clauses of $S$, then $S'$ is an expansion of
$S$ (all definitions in this section will be over fixed
$S$, and $S'$).
% formal defn:
\begin{defn}
  \label{defn:expansion}
  Let $f : \predof{S'} \to \predof{S}$ be such that
  % preserves arity:
  \textbf{(1)} for all $P' \in \predof{S'}$, $P'$ has the
  same vector of parameters as $f(P')$;
  % preserves clause structure:
  \textbf{(2)} for each clause $C' \in S'$, the clause $C$ (constructed
  by substituting all predicates $P'$ by $f(P')$) is in $S$.
  %
  Then $f$ is a \emph{correspondence} from $S'$ to $S$.
\end{defn}
%
If there is a correspondence from $S'$ to $S$, then $S'$ is an \emph{expansion}
of $S$, denoted $S \expandsto S'$.

%
\begin{defn}
  \label{defn:min-expansion}
  If $S'$ is CDD, $S \expandsto S'$, and there is no CDD system $S''$ such that $S \expandsto S' \expandsto S''$ and
  $S'' \neq S'$, then $S'$ is a minimal CDD expansion of $S$.
\end{defn}

% core algorithm
\begin{algorithm}[t]
  % Declare IO markers.
  \SetKwInOut{Input}{Input}
  %
  \SetKwInOut{Output}{Output}
  % Declare sub-program (procedure) markers.
  \SetKwProg{myproc}{Procedure}{}{}
  % Inputs: a heap program and an error location.
  \Input{For a recursion-free CHC system $S$.}
  % Output: inductive invariants.
  \Output{A solution to $S$ or $\none$.}
  % expand procedure:
  \myproc{$\sys(S)$ %
    \label{line:shara-begin}}{ %
    % construct expansion of C:
    $(S', \eta) \assign %
    \expand(S)$ \label{line:shara-expand} \; %
    % expand, solve, case split:
    \Switch{ $\solvecdd(S' )$ %
      \label{line:shara-case} }{
      % case: result is no solution:
      \lCase{ $\none$: }{ \Return{$\none$} \label{line:shara-ret-none} } %
      % case: result is a solution:
      \lCase{ $\sigma'$: }{ %
        \Return{ $\collapse{ \eta }{ \sigma' }$ } %
        \label{line:shara-ret-collapse}
      } %
    } %
  } %
  %
  \caption{\sys: a solver for recursion-free CHCs, which uses
    procedures $\expand{}$ (see \autoref{app:cons-cdd}) and
    $\solvecdd$ (see
    \autoref{sec:solve-cdd}). }
  \label{alg:shara}
\end{algorithm}
% walk through the sys algorithm:
\sys (\autoref{alg:shara}), given a recursion-free CHC system $S$
(\autoref{line:shara-begin}), 
returns a solution
to $S$ or the value $\none$ to denote that $S$ is
unsolvable.
%
\sys first runs a procedure $\expand$ on $S$ to obtain a CDD expansion
$S'$ of $S$.
%
\sys then invokes $\solvecdd$ on $S'$.
%
When \solvecdd~returns that $S'$ has
no solution, \sys~propagates $\none$ (\autoref{line:shara-ret-none}).
%
Otherwise, \sys constructs a solution from the CDD solution,
$\sigma'$, by invoking $\collapse{ \eta }{ \sigma' }$
(\autoref{line:shara-ret-collapse}).
%
$\textsc{Collapse}$ is designed to convert the solution for the CDD system
back to a sensible solution for the original problem. It does this by
taking the conjunction of all interpretations of predicates which
correspond to one predicate in the original problem.
%
That is, given a CDD solution $\sigma'$ and a correspondence $\eta$ from
$P' \in \predof{S'}$ to $P \in \predof{S}$,
$\collapse{\eta}{\sigma'}$ generates an entry in $\sigma$ for each
predicate in the original system:
$\sigma[P] \assign \bigland_{ \eta(P') = P} \sigma'(P')$.
%
$\expand$ is given in \autoref{app:cons-cdd}.

% shara correctness
\begin{thm}
  \label{thm:corr}
  $S$ is solvable if and only if $\shara$ returns solution $\sigma$.
\end{thm}
Proof is given in \autoref{app:corr}.

%%% Local Variables: 
%%% mode: latex
%%% TeX-master: "p"
%%% End: 
