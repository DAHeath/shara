% walk through the approach in technical detail
\section{Technical Approach}
\label{sec:approach}
%
This section presents the technical details of our approach.
%
\autoref{sec:CDD-defn} presents the class of Clause-Dependence Disjoint
systems and its key properties.
%
\autoref{sec:solve-cdd} describes how \sys solves CDD systems
directly.
%
\autoref{sec:core-solver} describes how \sys solves a given
recursion-free system by solving an equivalent CDD system.
%
Proofs of all theorems stated in this section are contained in appendix
\autoref{app:char} and \autoref{app:corr}.

\subsection{Clause-Dependence Disjoint Systems}
\label{sec:CDD-defn}
%
The key contribution of our work is the introduction of the class of
Clause-Dependence Disjoint (CDD) systems.
%
Solving an CHC system in CDD is coNP time.
%
CDD strictly contains
the union of classes of recursion-free systems that can be solved
in coNP time that have been introduced in previous work.
%
\begin{defn}
  \label{defn:cdds}
  For a given recursion-free CHC system $\cc{S}$,
  %
  if for all pair of uninterpreted predicates $P, Q \in \predof{S}$ that are siblings(\autoref{sec:chcs}), the 
  transitive dependences of $P$ and $Q$ are disjoint $\tdepsof{P} \cap \tdepsof{Q} = \emptyset$, and there is 
  no uninterpreted predicate shows more than once in a body of clause, then $\cc{S}$ is
  \emph{Clause-Dependence-Disjoint (CDD)}.
\end{defn}
%
CDD systems model hierarchical programs with branching and
procedure calls that in each execution path that each statments be
exected at most once.
%
\begin{ex}
  The CHC system $\mcchc$ (given in \autoref{sec:solve-ex}) is a CDD
  system.
  %
  The only clauses in $\mcchc$ that its body has multiple uninterpreted
  predicates are clause (7) that contains $\cc{L9}$ and $\cc{db}$.
  %
  The transitive dependences of $\cc{L9}$ and $\cc{db}$ are \{\cc{L6}\,\cc{L8}\,\cc{L4}\} and $\varnothing$, which are disjoint.
\end{ex}

% talk about body disjoint
The class of CDD systems contains tree structure system, and body-disjoint and linear
systems introduced in previous work.
%
Tree structure system is contained by body-disjoint system. 
%
For recursion-free system $S$, if for each uninterpreted predicate
$Q \in \predof{S}$, there is at most one caluse $C$ that $Q \in \bodyof{C}$,
and there is no uninterpreted predicate shows more than once in a body of clause, then
$S$ is \emph{body-disjoint}~\cite{rummer13a,rummer13b}.
%
If the body of each clause in $\cc{S}$ contains at most
relational predicate, then $\cc{S}$ is
\emph{linear}~\cite{albarghouthi12a}.


% talk about linear system
\begin{thm}
	\label{thm:cdd-contains}
  The class of CDD systems strictly contains the union of the classes
  of body-disjoint and linear systems.
\end{thm}
%
The CHC system $\mcchc$ (\autoref{sec:solve-ex}) is not body-disjoint
or linear, but is CDD.
%
A proof that both the classes of body-disjoint and linear systems are
contained in the class of CDD systems is given in \autoref{app:char}.

% describe how to solve a CDD system
\subsection{Solving a CDD system}
\label{sec:solve-cdd}

% algorithm for constructing a CDD system:
\begin{algorithm}[t]
  % Declare IO markers.
  \SetKwInOut{Input}{Input}
  %
  \SetKwInOut{Output}{Output}
  % Declare sub-program (procedure) markers.
  \SetKwProg{myproc}{Procedure}{}{}
  % Inputs: a heap program and an error location.
  \Input{A CDD System $\cc{S}$.}
  % Output: inductive invariants.
  \Output{If $\cc{S}$ is solvable, then a solution of
    $\cc{S}$; %
    otherwise, the value $\nosoln$.}
  % 
  \myproc{$\solvecdd(\cc{S})$ %
    \label{line:solve-begin}}{ %
    $\sigma \assign \emptyset$ \\
    $\cc{Preds} \assign \topSort(\predof{S})$ \\
    \For {$\cc{P} \in \cc{Preds}$}
    {$\cc{interpolant} \assign \solveitp(\prectr(\cc{P},\sigma),\postctr(\cc{P},\sigma))$\\
     \Switch{\cc{interpolant}}{
       \lCase{\cc{SAT}:}{\Return{$\nosoln$}}
       \lCase{$\cc{I}$:}{$\sigma$ [$\cc{P} \to \cc{I}$]}
     }
    }
    \Return{$\sigma$}
    }
  %
  \caption{$\solvecdd$: for a CDD system $\cc{S}$, returns a
    solution to $\cc{S}$ or the value $\none$ to denote that
    $\cc{S}$ has no solution.}
  \label{alg:solve-cdd}
\end{algorithm}


% introduce solveCDD
the procedure $\solvecdd$ (\autoref{alg:solve-cdd}) is a solver for CDD systems.
%

$\solvecdd$, given a CDD system $\cc{S}$, it first do a topological sort on all
uninterpreted predicates in $\cc{S}$ based on their dependency relations.
%
In the order of topological sort, for each uninterpreted predicates $\cc{P}$ 
 $\solvecdd$ calles $\solveitp$ to compute the binary interpolation of 
 encoded pre and post formulas with the current solution.
 %
 If the conjunction of pre and post formulas is satisifiable, then $\solvecdd$
 returns $\nosoln$ to indicate this CDD system $\cc{S}$ is not solvable.
 %
 Otherwise, $\solvecdd$ binds the uninterpreted predicate $\cc{P}$ to the
 interpolation result $\cc{I}$ as part of the soltuion.
 %
 The encoding procedure $\prectr$ and $\postctr$ are defined below.
 %
 $\solvecdd$ returns the soltuion $\sigma$. 

%
\begin{ex}
  $\solvecdd$, given $\mcchc$, may generate
  interpretations of $\mcpreds$ in any topological ordering of the
  dependency relation of $\mcchc$.
  %
  One such ordering is \cc{L4}, \cc{L6}, \cc{L8}, 
   \cc{L9}, \cc{db} and \cc{main}.
\end{ex}

\subsubsection{Constructing constrains of uninterpreted 
predicate}
%
For a CDD system $S$, given a uninterpreted predicate $\cc{P} \in \predof{S}$, and a partial 
solution $\sigma$ that maps some uninterpreted predicate to its solution formula, 
,the formula $\ctrof{P,\sigma}$ is a compact representation of constrains of $\cc{P}$ based on parital
solution $\sigma$, defined as follows.
% for each relational predicate, define constraint:
If partial solution $\sigma$ does not contains $P$, 
the constraint of $P$ under $\sigma$ that denoted as $\ctrof{P,\sigma}$ 
is constructed as: for all clauses $C \in S$ in which $P$ is the head of $\cc{C}$
\textbf{(1)} each predicate $Q \in \bodyof{C}$ is used and 
\textbf{(2)} the constrained of the clause $\consof{C}$ is encoded.
%
I.e., $\ctrof{P,\sigma}$ is
\[
\biglor_{ (\cc{C}_i \in \cc{S}) \land (\headof{C_i} = \cc{P})}
\left( \consof{C_i} \land %
\bigland_{ \cc{Q} \in \bodyof{C_i}} %
b_Q \right)
\]
%
If partial solution $\sigma$ contains $P$, then $\ctrof{P,\sigma}$ is the solution formula $\sigma[P]$.

% introduce the isused predicates:
For each $\cc{P} \in \predof{S}$, there is a corresponding boolean variable
$b_P$.
%
The counterexample characterization of $P$ constrains that
if $P$ is used, which means $b_{P}$ is \cc{True}, then its constraint must hold.
%
I.e., $\vc{P,\sigma}$ is $\neg b_{P} \lor \ctrof{P,\sigma}$.


\begin{ex}
  \label{ex:ctr}
  In the CHC system $\mcchc$ (given in \autoref{sec:solve-ex}), 
  when $\mcsolve$ solves the uninterpreted predicate \cc{L9}, it 
  has a partial solution $\sigma$ that maps \cc{L6} to $n \ge 0$.
  %
  For predicate \cc{L6}, the the $\ctrof{L6,\sigma}$ is $n \ge 0$.
  %
  For the 
  uninterpreted \cc{L9}, the $\ctrof{L9,\sigma}$ is $(abs' = n \land b_{L6}) \lor (abs' = n \land b_{L8})$.
  %
  The $\vc{L9,\sigma}$ is $\neg b_{L9} \lor \ctrof{L9,\sigma}$.
\end{ex}

% pre-constraint:
\subsubsection{Constructing the pre-constraint of a relational predicate}
\label{sec:cons-pre}
%
For a CDD system $\cc{S}$, given a uninterpreted predicate $\cc{P} \in \predof{\cc{S}}$
, because $\solvecdd$ solve each uninterpreted predicate in a toplogical order, 
it has a partial solution $\sigma : \cc{Q} \to \formula$ for all uninterpreted predicates in
$\depsof{P}$.
%
the pre-constraint $\prectr(\cc{P},\sigma)$ is a formula that the conjunction of the constrains
of $\cc{P}$ and solutions of all dependencies of $P$ with their boolean varialbe.
%
In particular, $\prectr(\cc{P},\sigma)$ is:
\[\ctrof{P,\sigma} \land
\left(
\bigland_{\cc{Q} \in \depsof{P}}
\left(\neg b_Q \lor \sigma[ \cc{Q} ]  \right)
\right)
\]
%
\begin{ex}
  \label{ex:pre-ctr}
  In the CHC system $\mcchc$ (given in \autoref{sec:solve-ex}), 
  when $\mcsolve$ solves the uninterpreted predicate \cc{L9}, because it solves
  in topological order, partial solution $\sigma$ contains \cc{L6} and \cc{L8}.
  %
  $\sigma$ maps
  \cc{L6} to $n \ge 0$ and \cc{L8} to $n < 0$.
  %
  The pre-formula for \cc{L9} under $\sigma$ is $\ctrof{L9} \land (\neg b_{L6} \lor n \ge 0) \land (\neg b_{L8} \lor n < 0)$.
  %
  The forumula of $\ctrof{L9,\sigma}$ is given in~\autoref{ex:ctr}.
\end{ex}

\subsubsection{Constructing the post-constraint of a relational
  predicate}
\label{sec:cons-ctx}
%
For $\sigma$ as a partial solution of
$S$ and $P \in \predof{S}$, the post-constraint $\postctr(P, \sigma)$ is mutually
inconsistent with the solution of $P$ in each solution of
$S$ that extends $\sigma$.
% dependents of R:
In particular, let $D_0$ be the reflexive transitive dependents of $P$
in $S$,
% clause siblings of dependents:
let $D_1$ be the siblings in $S$ of
$(D_0 \union {P})$,
%
let $D_2$ be all transitive dependences of $D_1$, and
% collect all predicates that we define a constraint over:
let $D = D_0 \union D_1 \union D_2$.
% define instantiation of C
The post constraint for $P$ under $\sigma$ is the conjunction of
counterexample characterization of all predicates $Q \in D$ and the query clause.
%
I.e., $\postctr(R, \sigma) =\cc{query} \land (\bigland_{\cc{Q} \in D} \vc{Q,\sigma})$.
% example post constraint:
\begin{ex}
  \label{ex:ctx-ctr}
  Consider the case in which $\mcsolve$, given $\mcchc$, generates an
  interpretation for \cc{L9}, introduced in \autoref{ex:pre-ctr}.
  %
  The reflexive transitive dependents $D_0$ of \cc{L9} is \{\cc{main}\}.
  %
  The siblings set $D_1$ is \{\cc{db}\}.
  %
  The transitive dependences of $D_1$ is $\varnothing$. 
  %
  Then the $D$ set is \{\cc{main},\cc{db}\}
  %
  The \cc{query} is $b_{main} \land res'<0$.
  %
  The post constraint for $L9$ under $\sigma$ is $\cc{query} \land \vc{main,\sigma} \land \vc{db,\sigma}$.
  %
  The whole formula is given in\QZ{add reference}. 
\end{ex}

\subsection{Solving recursion-free systems using CDD systems}
\label{sec:core-solver}

% give the general procedure:
$\shara$, given an recursion-free CHC system $S$,
constructs a CDD system $S'$ equivalent to $S$.
%
\sys then directly solves $S'$ and from its solution,
constructs a solution of $S$.
% define an expansion of a CHC system:
For two given recursion-free CHC system $S$ and $S'$, if
there is a homomorphism from $\predof{S'}$ to $\predof{S}$ that
preserves the relationship between the clauses of $S'$ in
the clauses of $S$, then $S'$ is an expansion of
$S$ (all definitions in this section will be over fixed
$S$, and $S'$).
% formal defn:
\begin{defn}
  \label{defn:expansion}
  Let $f : \predof{S'} \to \predof{S}$ be such that
  % preserves arity:
  \textbf{(1)} for all $P' \in \predof{S'}$, $P'$ has the
  same vector of parameters as $f(P')$;
  % preserves clause structure:
  \textbf{(2)} for all clause $C' \in S'$, the new clause $C$
  by substituting all predicates $P'$ by $f(P')$ is in $S$.
  %
  Then $f$ is a \emph{correspondence} from $S'$ to $S$.
\end{defn}
%
If there is a correspondence from $S'$ to $S$,
$S'$ is an \emph{expansion} of $S$, denoted
$S \expandsto S'$.
%

% core algorithm
\begin{algorithm}[t]
  % Declare IO markers.
  \SetKwInOut{Input}{Input}
  %
  \SetKwInOut{Output}{Output}
  % Declare sub-program (procedure) markers.
  \SetKwProg{myproc}{Procedure}{}{}
  % Inputs: a heap program and an error location.
  \Input{For a recursion-free CHC system $S$.}
  % Output: inductive invariants.
  \Output{A solution to $S$ or $\none$.}
  % expand procedure:
  \myproc{$\sys(S)$ %
    \label{line:shara-begin}}{ %
    % construct expansion of C:
    $(S', \eta) \assign %
    \expand(S)$ \label{line:shara-expand} \; %
    % expand, solve, case split:
    \Switch{ $\solvecdd(S' )$ %
      \label{line:shara-case} }{
      % case: result is no solution:
      \lCase{ $\none$: }{ \Return{$\none$} \label{line:shara-ret-none} } %
      % case: result is a solution:
      \lCase{ $\sigma'$: }{ %
        \Return{ $\collapse{ \eta }{ \sigma' }$ } %
        \label{line:shara-ret-collapse}
      } %
    } %
  } %
  %
  \caption{\sys: a solver for recursion-free CHCs, which uses
    procedures $\expand{}$ (see \autoref{app:cons-cdd}) and
    $\solvecdd$ (see
    \autoref{sec:solve-cdd}). }
  \label{alg:shara}
\end{algorithm}
% walk through the sys algorithm:
$\shara$ (\autoref{alg:shara}), given an recursion-free CHC system $S$ (\autoref{line:shara-begin}), 
returns a solution
to $S$ or the value $\none$ to denote that $S$ is
unsolvable.
%
$\shara$ first runs a procedure $\expand{}$ on $S$ to obtain a CDD expansion $S'$ (\autoref{defn:min-cdd-expansion})
of $S$.
%
$\shara$ then runs a procedure $\solvecdd$ on $S'$, which either returns the value
$\none$ to denote that $S'$ has no solution or a solution
$\sigma'$ (\autoref{line:shara-case}).
%
If $\solvecdd$ returns that $S'$ has
no solution, then \sys returns that $S$ has no solution
(\autoref{line:shara-ret-none}).
%
Otherwise, if $\solvecdd$ returns a solution
$\sigma'$, then $\sys$ returns, as a solution to $S$,
$\collapse{ \eta }{ \sigma' }$, defined below
(\autoref{line:shara-ret-collapse}).

% define collapse:
For $\sigma'$ and %
correspondence $\eta$ from $P' \in \predof{S'}$ to $P \in \predof{S}$,
$\collapse{\eta}{\sigma'}$ maps solution $\sigma'$ to
$\sigma$[$P \to \bigland_{ \eta(P') = P}
\sigma'(P')$].
%
$\expand{ \mathcal{R} }$ is given in \autoref{app:cons-cdd}.

% shara correctness
\begin{thm}
  \label{thm:corr}
  $S$ is solvable if and only if $\shara$ returns solution $\sigma$.
\end{thm}

%%% Local Variables: 
%%% mode: latex
%%% TeX-master: "p"
%%% End: 
