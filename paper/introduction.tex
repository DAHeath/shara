% intro:
\section{Introduction}
\label{sec:intro}
% context, problem defn:
Many critical problems in program verification, such as the
verification of safety properties of recursive programs and
concurrent programs, can be reduced to solving systems of
\emph{Constrained Horn Clauses} (CHCs), a class of logic-programming
problems~\cite{bjorner13,flanagan03,rummer13a,rummer13b}.


% general introduce chc system
A CHC consists of a body of applications of uninterpreted relational
predicates, a constraint in a first-order theory, and a head
application of a relational-predicate.
%
A CHC system consists of a set of CHCs and a query relational
predicate.
% 
A solution of a system is an interpretation of each relational
predicate as a formula such that each clause body with predicate
symbols substituted with their interpretation entails the interpreted
clause head.
\QZ{This seems really out of place -- DAH}

% introduce the problem of solving recursion-free systems:
In this work we focus on the subclass of CHC systems which are known
as \emph{recursion-free}. In a recursion-free CHC system, the
derivation of each relational predicate will contain no references to
itself. \QZ{\DAH{I still think we need a small example here}} I.e., no
relational predicate is \emph{dependent} on itself.
%
Recursion-free CHC systems are an important subclass for two reasons.
%
First, recursion-free systems can be used to model safety properties
for hierarchical programs~\cite{lal-qadeer15,lal-qadeer-lahiri12},
i.e., programs with control branches and procedure calls, but only
bounded iteration and recursion.
%
Second (and most importantly), the task of solving a general CHC
system can reduce to the task of solving a sequence of recursion-free
systems.
%
Many \QZ{more references? Or remove many.} modern solvers generate
such a sequence where each subsystem is a bounded unwinding of the
input system. These solvers attempt to show that the solution to a
given subsystem generalizes to the original system or else generate a
larger subsystem for the next iteration~\cite{bjorner13}.
%
Such solvers are heavily reliant on the performance of the underlying
recursion-free solver since they invoke this underlying solver in an
iterative loop.
%
% current general techniques for solving recursion-free systems
Typically, even recursion-free CHC systems are not solved directly.
Instead, they are reduced to some even more specific subclass of CHC
system. For example, some approaches generate CHC systems whose
derivations are trees~\cite{bjorner13}.
\QZ{In general it seems we need many more references in the intro.}
%
A solver can dispatch such a system by issuing interpolation queries
to find a suitable definition for each relational predicate.
%
We refer to the subclass of CHC systems that can be solved by such
techniques as \emph{directly solvable}.

% Time complexity
In general, solving a recursion-free CHC system whose facts are
restricted to the theory of integer linear arithmetic is
co-NEXPTIME-complete~\cite{rummer13b}.
%
In contrast, solving a directly solvable system in the same theories
is co-NP~~\cite{rummer13b}.
%
Therefore, the performance of a CHC solver is highly reliant on the
order (i.e the number of predicates) of the directly solvable systems
it produces.

% current subclass
Previous work has introduced three classes of directly solvable
systems,
% derivation trees:
\textbf{(1)} tree systems~\cite{heizmann10,bjorner13,mcmillan14}, %
% disjunctive trees:
\textbf{(2)} body-disjoint systems, which represent multiple
derivation trees with a single disjunctive tree
system~\cite{rummer13a,rummer13b}, and
% linear systems:
\textbf{(3)} linear CHC systems, which can compactly represent
multiple linear derivations with a single system structured as a
Directed Acyclic Graph (DAG)~\cite{albarghouthi12a}.
%
The class of body-disjoint systems strictly contains the class of
derivation trees, and is independent of the class of linear systems
(i.e., the classes overlap, and neither contains the
other)~\cite{rummer13a,rummer13b}.

% contribution of this paper: CDD systems:
The first contribution of this paper is the introduction of a novel
class of directly solvable systems that is a superset of both
the body-disjoint subclass and the linear subclass.
%
In particular, the class consists of all recursion-free CHC systems
such that, for each clause $C$, the dependences of all distinct
relational predicates in the body of $C$ are disjoint \DAH{\QZ{Should
we illustrate what disjoin means here?}};
%
we thus refer to such systems as \emph{Clause-Dependent Disjoint
(CDD)} systems.
%
We demonstrate that the class of CDD systems is directly solvable
through a symbolic algorithm.
%

% a new solver
The second contribution of this paper is a solver for CHC systems,
named \sys.
%
Given a recursion-free system $S$, \sys reduces the problem
of solving $S$ to solving an equivalent CDD system
$S'$.
%
In the worst case, it is possible that the order of $S'$ may be
exponential in the order of $S$.
%
However, empirically we have found that the order of $S'$ is usually
close enough to the order of $S$ that \sys often outperforms the best
known CHC solvers.
%
Because the class of CDD systems is a superset of both body-disjoint
and linear systems, the order of an equivalent CDD system is always
less than or equal to that of systems in these subclasses.
%
The procedure implemented in \sys to construct $S'$ is a
generalization of existing procedures for generating compact
verification conditions of hierarchical
programs~\cite{flanagan01,lal-qadeer15}.
%
Given a general (possibly recursive) CHC system, \sys solves a sequence of
recursion-free systems.
%
Each subsystem is a bounded unwinding of the original system. \sys
attempts to combine the solutions of these recursion-free systems to
synthesize a soltuion of the original solution, as has been proposed
in previous work~\cite{rummer13b}.

% experience:
\QZ{need to change if need}
We implemented \sys within the \duality CHC solver~\cite{bjorner13},
which is implemented within the \zthree automatic theorem
prover~\cite{moura08}.
%
We evaluated the effectiveness of \sys on standard benchmarks drawn
from SVCOMP15~\cite{svcomp15}.
%
The results indicate that in a strong majority of practical cases,
\sys performs better than modern solvers.
%
Futhermore, the results indicate that combining the strengths of \sys
with that of other existing approaches (as discussed in
\autoref{sec:evaluation}) is a promising direction for the future of
CHC solving.

% paper outline:
The rest of this paper is organized as follows.
%
\autoref{sec:overview} illustrates the operation of \sys on a
recursion-free CHC system.
%
\autoref{sec:background} reviews technical work on which \sys is
based.
%
\autoref{sec:approach} describes \sys in technical detail.
%
\autoref{sec:evaluation} gives the results of our empirical evaluation
of \sys.
%
\autoref{sec:related-work} compares \sys to related work.

%%% Local Variables: 
%%% mode: latex
%%% TeX-master: "p"
%%% End: 
