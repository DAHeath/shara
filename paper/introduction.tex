% intro:
\section{Introduction}
\label{sec:intro}
% context, problem defn:
Many critical problems in program verification, such as the
verification of safety properties of sequential recursive programs and
concurrent programs, can be reduced to solving systems of
\emph{Constrained Horn Clauses} (CHCs), a class of logic-programming
problems~\cite{bjorner13,flanagan03,rummer13a,rummer13b}.


% general introduce chc system
A CHC consists of a body of applications of uninterpreted relational
predicates, a constraint in a first-order theory, and a head
application of a relational-predicate.
%
A CHC system consists of a set of CHCs and a query relational
predicate.
% 
A solution of a system is an interpretation of each relational
predicate as a formula such that each clause body with predicate
symbols substituted with their interpretation entails the interpreted
clause head.

% introduce the problem of solving recursion-free systems:
Recursion-free CHC systems, in which each
relational-predicate does not occur in a derivation (i.e., is not
a \emph{dependency}) of itself, is an important subclass of general
CHC systems for two reasons.
%
First, many modern solvers solves general CHC systems by solving 
a series of recursion-free systems from bounded unwindings of given
recursive CHC system and combining their solutions~\cite{bjorner13}.
%
The performance of the solver for recursive systems critically depends
on the performance of the solver for recursion-free systems that it
uses. 
%
Second, solving recursion-free systems formulates proving safety of
hierarchical programs~\cite{lal-qadeer15,lal-qadeer-lahiri12}, i.e.,
programs with control branches and procedure calls, but only bounded
iteration and recursion.

% current general techniques for solving recursion-free systems
Many efficient CHC solvers, solve a recursion-free CHC system 
by reducing it to one or a series of subclass of 
recursion-free CHC system.
%
Then solvers solve these subclass of recursion-free CHC systems by
issuing an interpolation query for each relational predicate.
%
We refer to such subclass of systems as \emph{directly solvable}.
%

% state the complexity of solving general problem,and explain the 
% why it is fast
In general, solving a recursion-free CHC system for the theory of
integer linear arithmetic is co-NEXPTIME-complete~\cite{rummer13b}.
%
In contrast, solving a directly solvalbe system for the theory of
integer linear arithmetic is co-NP~~\cite{rummer13b}.
%
Therefore, the performance of a CHC solver for a recursion-free CHC system 
typically is directly determined by the size(i.e the number of predicates) 
of the subclass systems it reduces to.

% current subclass
Previous work has introduced three classes of directly solvable systems,
% derivation trees:
\textbf{(1)} tree structure systems~\cite{heizmann10,bjorner13,mcmillan14}, %
% disjunctive trees:
\textbf{(2)} body-disjoint systems, which represent multiple
derivation trees with a single disjunctive tree
system~\cite{rummer13a,rummer13b}, and
% linear systems:
\textbf{(3)} linear CHC systems, which can compactly represent
multiple linear derivations with a single system structured as a
Directed Acyclic Graph (DAG)~\cite{albarghouthi12a}.
%
The class of body-disjoint systems strictly contains the class of
derivation trees, and is independent of the class of linear systems
(i.e., the classes overlap, and neither contains the
other)~\cite{rummer13a,rummer13b}.


% contribution of this paper: CDD systems:
The first contribution of this paper is the introduction of a novel
class of directly solvable systems that strictly
contains the union of the classes of body-disjoint and linear systems.
%
In particular, the class consists of all CHC systems such that, for
each clause $\mathcal{C}$, the dependences of all distinct relational
predicates in $\mathcal{C}$ are disjoint;
%
we thus refer to such systems as \emph{Clause-Dependent Disjoint
  (CDD)} systems.
%
We present a symbolic algorithm that demonstrates that CDD is directly solvable.
%

% a new solver
The second contribution of this paper is a novel solver for
CHC systems, named \sys.
%
Given recursion-free system $\mathcal{S}$, \sys reduces the problem
of solving $\mathcal{S}$ to solving an equivalent CDD system
$\mathcal{S}'$.
%
In the worst case, the size of $\mathcal{S}'$ may be exponential in the size of
$\mathcal{S}$.
%
In our experience, the size of $\mathcal{S}'$ is usually close enough
to the size of $\mathcal{S}$ for \sys to perform significantly better
than the best known CHC solvers.
%
Because CDD strictly contains the union of the classes 
of body-disjoint and linear systems, the size of an equivalent minimum 
CDD system is equal or less then the size of equivalent body-disjoint and
linear system.
%
The procedure implemented in \sys to construct $\mathcal{S}'$ is a
generalization of existing procedures for generating compact
verification conditions of hierarchical
programs~\cite{flanagan01,lal-qadeer15}.
%
Given a recursion system, \sys solves a series recursion-free systems 
from bounded unwinding the original system and combines the solutions
of recursion-free systems to generate the solution for the recursion system
as previous work proposed~\cite{rummer13b}.
%

% experience:
\QZ{need to change if need}
We implemented \sys within the \duality CHC solver~\cite{bjorner13},
which is implemented within the \zthree automatic theorem
prover~\cite{moura08}.
%
We evaluated the effectiveness of \sys on standard benchmarks drawn
from SVCOMP15~\cite{svcomp15}.
%
The results indicate that in a strong majority of practical cases,
\sys performs better than modern solvers.
%
The results indicate that solvers that perform better than both modern
solvers and \sys could be designed by combining the strengths of both
approaches (discussed in \autoref{sec:evaluation}).

% paper outline:
The rest of this paper is organized as follows.
%
\autoref{sec:overview} illustrates the operation of \sys using an
example hierarchical program and corresponding CHC system.
%
\autoref{sec:background} reviews technical work on which \sys is
based.
%
\autoref{sec:approach} describes \sys in technical detail.
%
\autoref{sec:evaluation} gives the results of our empirical evaluation
of \sys.
%
\autoref{sec:related-work} compares \sys to related work.

%%% Local Variables: 
%%% mode: latex
%%% TeX-master: "p"
%%% End: 
