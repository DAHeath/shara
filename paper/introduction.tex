% intro:
\section{Introduction}
\label{sec:intro}
% context, problem defn:
Many critical problems in program verification can be reduced to
solving systems of Constrained Horn Clauses (CHCs), a class of
logic-programming
problems~\cite{bjorner13,flanagan03,rummer13a,rummer13b}.
%
A CHC is a logical implication with the following form:
$$
  R_1(\vec{v_1}) \leftarrow R_2(\vec{v_2}) \land R_3(\vec{v_3}) \land
  ... \land \varphi(\vec{v_0}, \vec{v_1}, \vec{v_2}, \vec{v_3},...)
$$
Here, the left side of the implication, called the head, contains an
uninterpreted relational predicate applied to a vector of variables.
%
The right side has any number of such predicates conjoined together
with a \emph{constraint} ($\varphi$). The constraint is a logical formula in a background
theory and may use variables named by the predicates.
%
A CHC system is a set of CHCs.
%
The goal of the CHC solving problem is to find suitable interpretations
for each predicate such that each CHC is logically
consistent in isolation.

% introduce the problem of solving recursion-free systems:
In this work we focus on the subclass of CHC systems which are known
as \emph{recursion-free}. In a recursion-free CHC system, no
derivation of a predicate will invoke that predicate.
%
Less formally, a recursion-free CHC system is one where following
implication arrows through the system will never reach the same clause
twice.
%
Recursion-free CHC systems are an important subclass for two reasons.
%
First, recursion-free systems can be used to model safety properties
for hierarchical programs~\cite{lal-qadeer15,lal-qadeer-lahiri12}
(programs with only bounded iteration and recursion).
%
Second and most importantly, a well-known approach for solving a
general CHC system reduces the input problem to solving a sequence
of recursion-free systems.
%
Such approaches attempt to synthesize a solution for the   original
system from the solutions of recursion-free systems~\cite{bjorner13}.
%
The performance of such solvers relies
heavily on the performance of solving recursion-free CHC systems.
%

% current general techniques for solving recursion-free systems
Typically, even recursion-free CHC systems are not solved directly.
%
Instead, they are reduced to a more specific subclass of
recursion-free CHC system.
%
These classes include those of
\emph{body-disjoint} (or \emph{derivation tree})
systems~\cite{heizmann10,bjorner13,mcmillan14,rummer13a,rummer13b} and
of \emph{linear} systems~\cite{albarghouthi12a}.
%
We will discuss these classes in \autoref{sec:overview} and
\autoref{sec:related-work}.
%
Such classes can be solved by issuing
\emph{interpolation queries} to find suitable definitions for the
uninterpreted predicates.
%

% Time complexity
In general, solving a recursion-free CHC system for
propositional logic and the theory of linear integer arithmetic is
co-NEXPTIME-complete~\cite{rummer13b}.
%
In contrast, solving a linear system or body-disjoint system with the
same logic and theories is in co-NP~\cite{rummer13b}.
%
We refer to such classes that are solvable in co-NP time as
\emph{directly solvable}.
%
Because solving an arbitrary recursion-free system is harder than
solving a directly solvable system, solvers which reduce to directly
solvable systems are highly reliant on the size of the reductions.

% contribution of this paper: CDD systems:
The first contribution of this paper is the introduction of a novel
class of directly solvable systems that we refer to as
\emph{Clause-Dependence Disjoint} (CDD).
%
The formal definition of CDD is given at ~\autoref{defn:cdds}.
%
CDD is a strict superset of the union of previously introduced classes
of directly solvable systems.
%
The key characteristic of this class is that when an arbitrary
recursion-free system is reduced to a CDD system and to a system from
a different directly solvable class, the CDD system is frequently the
smaller of the two.
%
Therefore, solving recursion-free systems by reducing them to CDD form
is often less computationally expensive than reducing them to a
system in a different class.

% a new solver
The second contribution of this paper is a solver for CHC systems,
named \sys.
%
Given a recursion-free system $S$, \sys reduces the problem of solving
$S$ to solving a CDD system $S'$.
%
In the worst case, it is possible that the size of $S'$ may be
exponential in the size of $S$.
%
However, empirically we have found that the size of $S'$ is usually
close enough to the size of $S$ that \sys frequently outperforms
\duality, one of the best known CHC solvers.
%
The procedure implemented in \sys is a generalization of existing
techniques that synthesize compact verification conditions for
hierarchical programs~\cite{flanagan01,lal-qadeer15}.
%
Given a general (possibly recursive) CHC system, \sys solves a
sequence of recursion-free systems.
%
Each subsystem is a bounded unwinding of the original system. \sys
attempts to combine the solutions of these recursion-free systems to
synthesize a solution to the original problem, as has been proposed
in previous work~\cite{rummer13b}.

% experience:
We implemented \sys within the \duality CHC solver~\cite{bjorner13},
which is implemented within the \zthree automatic theorem
prover~\cite{moura08}.
%
We evaluated the effectiveness of \sys on standard benchmarks drawn
from SVCOMP15~\cite{svcomp15}.
%
The results indicate that \sys outperforms modern solvers many cases.
%
Futhermore, the results indicate that combining the strengths of \sys
with that of other existing approaches (as discussed in
\autoref{sec:evaluation}) is a promising direction for the future of
CHC solving.

% paper outline:
The rest of this paper is organized as follows.
%
\autoref{sec:overview} illustrates the operation of \sys on a
recursion-free CHC system.
%
\autoref{sec:background} reviews technical work on which \sys is
based.
%
\autoref{sec:approach} describes \sys in technical detail.
%
\autoref{sec:evaluation} gives the results of our empirical evaluation
of \sys.
%
\autoref{sec:related-work} compares \sys to related work.
