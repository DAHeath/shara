\documentclass{article}

\usepackage{relsize}
\usepackage{amsmath}

\newcommand{\cc}[1]{\mbox{\smaller[0.5]\texttt{#1}}}

\newcommand{\DUALITY}{\textsc{Duality}}
\newcommand{\SHARA}{\textsc{Shara}}
\newcommand{\RHORT}{\textsc{Rhort}}

\begin{document}
When \DUALITY~attempts to interpolate a DAG, it is forced to unwind the DAG to a
tree.
%
When this DAG happens to have been constructed from complex proof rules (such
as in the case of Relational, Higher-Order Refinement Type Inference) this tree
will probably be far too large for \DUALITY~to solve in a reasonable amount of
time.
%
There are three sources of inefficiency which we can fix with a more
appropriate solver for this type of problem.

\begin{enumerate}
\item The solver need not unwind the DAG to a tree, but rather only a disjoin
dependency DAG (\SHARA). The importance of this optimization is greatly
exaggerated by proof systems that include a context, since these contexts often
end up being used across various branches of the same proof.
%
This means that it is very likely that \SHARA~could reduce the number of
required interpolants exponentially compared to \DUALITY~in most cases.
%
\item The Solver need not calculate every interpolant. The reason for this is
quite simple: We \emph{can't} use certain interpolants to help us construct a
proof. So there is no reason to calculate an interpolant there in the first
place. The biggest source of this is, again, contexts. In \RHORT, every Lambda
and Match expression spawns new context vertices. However, none of the facts at
these interpolants can be used to help generalize a proof.
%
\item The interpolants need not be solved in any particular order. The approach
taken in \SHARA~was to solve the interpolants in topological order. However,
this is not essential, merely convenient. A more appropriate strategy would be
to try to maximally reduce the number of dependencies in the DAG at each step.

\end{enumerate}

I propose a solver which takes advantage of these three facts. First, I propose
a modification to the solver's type. \DUALITY~can be seen as having the following type:

\begin{align*}
\DUALITY : \mathsf{CHC} \rightarrow \mathsf{Query} \rightarrow (\mathsf{Model} + \mathsf{Proof})
\end{align*}

\DUALITY~takes a (potentially recursive) CHC system and a query, and finds a
counterexample model or a valid proof. \SHARA~should instead have the following type:

\begin{align*}
\SHARA &: \mathsf{ProofStructure} \\
  &\rightarrow \mathsf{NonrecCHC} \\
  &\rightarrow \mathsf{Query} \\
  &\rightarrow (\mathsf{Model} + \mathsf{PartialProof} + \mathsf{Proof})
\end{align*}

In addition to asking the CHC system to be nonrecursive, this new signature
asks the user to provide a structure for the current proof. This structure
contains the information needed to construct entailments whose validity would
generalize a hierarchical proof.
%
\SHARA~can use this information to 1) reduce the number of interpolant
calculations and to 2) parallelize computing entailments of the interpolants.
%
\SHARA~has three possible terminating outcomes.
\begin{enumerate}
\item A counterexample model was found.
\item A proof of the hierarchical program was found, but it did not generalize.
In this case, the partial proof will indicate which interpolants generalized
and which did not so that the user can intelligently restrict construction of
their next hierarchical approximation.
\item A proof of the hierarchical program was found, and it generalized. This
would indicate the user is now done with their proof entirely.
\end{enumerate}

The operation of \SHARA~is to be a parallel, recursive algorithm which at each
step attempts to reduce the number of dependencies in the DAG. At any point, if
there are no \emph{relevant} interpolants left to calculate, that branch of the
computation can be killed. An interpolant is relevant if it appears somewhere
in the proof structure.

\SHARA~works in the following way: First, it constructs a disjoint dependency
DAG based on the input DAG. It then performs the following recursive algorithm:

\begin{enumerate}
\item \SHARA~calculates a \emph{minimum cut} of the current subgraph. The
minimum cut is calculated by assigning each vertex in the subgraph a weight
inversely related to the number of dependencies which cross that vertex. Two
vertices are dependent on one another if they exist on the same hyperpath in
the DAG. This cut partitions the subgraph into three pieces: The vertices which
must be interpolated \emph{now} to perform the cut, and the two new subgraphs
on either side of this cut.
\item \SHARA~interpolates the vertices along the cut before recursively
dispatching the two subgraphs \emph{in parallel}.
\item \SHARA~combines the resulting interpolant solutions from the cut and the
two parallel calls.
\item If the current invocation has the solution for all instances of a
relevant vertex from the original DAG, it combines those solutions to construct
an interpolant for the original DAG.
\item If the current invocation has the solution to both relevant vertices
involved in an entailment in the proof structure, it can query the entailment
before returning.
\end{enumerate}

In addition to this, the following optimizations can be made:
\begin{enumerate}
\item The proof structure can contain distinct, optional ways to complete the
proof tree at each level. If \SHARA~finds that one of these options is valid,
the remaining options can be removed, potentially reducing the number of
relevant interpolants.
\item If the current subgraph contains only a small number of relevant
interpolants (below some threshold), it can calculate them all sequentially
rather than continuing to cut. This could reduce the number of interpolants to
calculate since each cut might contain irrelevant interpolants.
\end{enumerate}

In short, the purpose of this algorithm is to try to gain two advantages over
\DUALITY: 1) To aggresively reduce the number of required interpolant
calculations 2) To parallelize SMT calls (both interpolation and and
entailment) as much as possible.
\end{document}
